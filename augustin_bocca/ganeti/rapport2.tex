\Chapter {Finalement installer ganeti et xen}

Pour installer Ganeti et xen sur une meme machine il faut rester en squeeze pour xen et installer Ganeti de testing.

pour commencer, éditer le fichier /etc/apt/sources.list.:

Ajouter les depots qui vous interesse :

################################################
## squeeze
deb http://ftp.fr.debian.org/debian/ squeeze main contrib non-free
deb-src http://ftp.fr.debian.org/debian/ squeeze main contrib non-free

## squeeze multimedia
deb http://www.debian-multimedia.org squeeze main non-free
deb-src http://mirror.home-dn.net/debian-multimedia squeeze main

# squeeze security
deb http://security.debian.org/ squeeze/updates main contrib non-free
deb-src http://security.debian.org/ squeeze/updates main contrib non-free

# squeeze update
deb http://ftp.fr.debian.org/debian/ squeeze-updates main contrib non-free
deb-src http://ftp.fr.debian.org/debian/ squeeze-updates main contrib non-free

################################################
## wheezy
deb http://ftp.fr.debian.org/debian/ wheezy main contrib non-free
deb-src http://ftp.fr.debian.org/debian/ wheezy main contrib non-free

## wheezy multimedia
deb http://www.debian-multimedia.org wheezy main non-free
deb-src http://www.debian-multimedia.org/ wheezy main

## wheezy security
deb http://security.debian.org/ wheezy/updates main contrib non-free
deb-src http://security.debian.org/ wheezy/updates main contrib non-free

################################################
## sid
deb http://ftp.fr.debian.org/debian/ sid main contrib non-free
deb-src http://ftp.fr.debian.org/debian/ sid main contrib non-free

## sid multimedia
deb http://www.debian-multimedia.org/ sid main
deb-src http://www.debian-multimedia.org/ sid main

################################################
## experimental
deb http://ftp.fr.debian.org/debian/ experimental main contrib non-free
deb-src http://ftp.fr.debian.org/debian/ experimental main contrib non-free

# experimental multimedia (uniquement 64 bits, décommentez les lignes dans ce cas)
#deb http://www.debian-multimedia.org/ experimental main
#deb-src http://www.debian-multimedia.org/ experimental main 


on va ensuite dans le répertoire /etc/apt/apt.conf.d et on crée un fichier 80default-distrib (le nom du fichier est libre). On y place l’instruction suivante:

APT::Default-Release "stable";

Nous y sommes. Avec la configuration ci-dessus, on reste par défaut sur la version stable lors des mises à jour. Mais on a la possibilité d’installer un paquet testing très simplement.

Tout d’abord faire un update d’apt:

apt-get update

puis installer le paquet voulu grâce à -t testing:

apt-get -t testing install mon_paquet
