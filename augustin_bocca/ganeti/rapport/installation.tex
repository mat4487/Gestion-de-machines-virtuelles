\section {Installation}
\subsection {Modification des sources}
Nous avons intallé ganeti à partir de la branche testing de debian. Pour des raisons techniques le système est squeeze. Pour cela il faut ajouter les sources de testing dans le fichier /etc/apt/sources.list :

##Wheezy
deb http://ftp.fr.debian.org/debian/ wheezy main contrib non-free
deb-src http://ftp.fr.debian.org/debian/ wheezy main contrib non-free

## wheezy security
deb http://security.debian.org/ wheezy/updates main contrib non-free
deb-src http://security.debian.org/ wheezy/updates main contrib non-free


IL faut ensuite créer le fichier de préférence de apt dans le répertoire /etc/apt/apt.conf.d. Nous avons appelé le  fichier 80default-distrib (le nom du fichier est libre). Il faut ajouter cette ligne au fichier qui défini la branche stable comme la branche par défaut :

APT::Default-Release "stable";

\subsection {Mise à jour et installation}
On peut enfin alors mettre le système à jour et installer ganeti :

apt-get update && apt-get dist-upgrade -q -y --force-yes
apt-get -t testing install -q -y --force-yes ganeti2 ganeti-htools ganeti-instance-debootstrap

\section {Configuration}

La configuration est l'étape la plus complexe.

\subsection {Configuration du fichier hosts}

Dans le fichier hosts il faut renseigner l'adresse et le nom complet du node primaire de cette manière :

172.16.68.10    griffon-10.nancy.grid5000.fr griffon-10

iśubsection {Copie de vmlinuz-2.6.32-5-xen-amd64 et initrd.img-2.6.32-5-xen-amd64}

- Dans /etc/boot copier les fichiers vmlinuz-2.6.32-5-xen-amd64 et initrd.img-2.6.32-5-xen-amd64 :

cp vmlinuz-2.6.32-5-xen-amd64 vmlinuz-2.6-xenU
cp initrd.img-2.6.32-5-xen-amd64 initrd.img-2.6-xenU

\subsection {Création du bridge xen-br0}

\subsection {Création du LVM}

Ganeti requière un LVM d'au moins 20Go pour fonctionner.

Sur les neud de grid5000 il est possible de créer un tel LVM sur la partition /dev/sd5.

umount /dev/sda5
pvcreate /dev/sda5
vgcreate xenvg /dev/sda5

On a créé un VG qui se nome xenvg sur /dev/sda5.

\subsection {Edition de /usr/share/ganeti/os/debootstrap/common.sh}

Il est nécessaire d'éditer ce fichier pour que ganeti puisse créer des instances :

{Fichier de conf_common}

Par defaut le mirroir utilisé par ganeti est http://ftp.us.debian.org/debian/. Sur Grid5000 les depots US sont bloqués. Il faut donc indiquer les depots francais.
On indique aussi l'adresse du  proxy de grid5000.
On peut choisir la version de Debian que l'on souhaite installer. Ici nous avons opter pour squeeze.
L'architecture que nous choisi est amd64, car l'hote la supporte.
La variable : EXTRA_PKGS permet d'installer des paquets supplémentaires.

\subsection {Configuration et initialisation du cluster}
L'initialisation du cluster se fait avec la commande gnt-cluster init clusterX

#initialisation du cluster
gnt-cluster init --no-drbd-storage --nic-parameters link=eth0 cluster1

Ici nous avons précisé les options --no-drbd-storage --nic-parameters link=eth0.
La première permet d'utiliser ganeti sans utiliser la haute disponibilité.
La seconde permet de préciser un autre bridge que xen-br0

Enfin il faut renseigner le inird et le root_path, cela est nécessaire pour la création des instances :

gnt-cluster modify --hypervisor-parameter xen-pvm:initrd_path='/boot/initrd.img-2.6-xenU'
gnt-cluster modify --hypervisor-parameter xen-pvm:root_path='/dev/xvda1'

