\Chapter {Finalement installer ganeti et xen}

Pour installer ganeti et xen sur un même node il faut rester en squeeze pour xen et installer ganeti à partir  de la branche testing.

pour commencer, éditer le fichier /etc/apt/sources.list.:

Ajouter les depots de wheezy :

##Wheezy
deb http://ftp.fr.debian.org/debian/ wheezy main contrib non-free
deb-src http://ftp.fr.debian.org/debian/ wheezy main contrib non-free

## wheezy security
deb http://security.debian.org/ wheezy/updates main contrib non-free
deb-src http://security.debian.org/ wheezy/updates main contrib non-free


On va ensuite dans le répertoire /etc/apt/apt.conf.d et on crée un fichier 80default-distrib (le nom du fichier est libre). On ajoute la  suivante:

APT::Default-Release "stable";

On peut ensuite installer ganeti avec :
apt-get -t testing install ganeti2

\section {ganeti.sh}

ganeti.sh permet d'automatiser l'intallation de ganeti sur une debian squeeze.
Le script effectue l'opération précédente ainsi que la configuration du fichier /etc/network/interfaces dans le quel on définie le bridge de xen xen-br0 et la création du LVM de 20Go nécessaire pour ganeti.
Le script configure aussi les fichiers hosts et hostname.
Et enfin il initialise le cluster de ganeti et il ajoute le node master.

\section {vm.sh}

Le sript vm.sh créer des machines virtuelles xen sur un noeud.
Tout d'abort le script créer le fichier /etc/modprobe.d/xen qui permet de créer suffisament de loop pour les machines virtuelles.
Ensuite il créer un Volume Logique dans le LVM créer pour ganeti. Dans ce LV seront stockées les machines virtuelles.
Puis les différents fichiers de configurations des machines virtuelles (/etc/xen/domUX) sont créés en leurs attribuant des adresses IP locale (10.0.0.1 à 10.0.0.7) et des adresses MAC aléatoires.
Ensuite les fichiers  /opt/xen/domains/domU1/disk.img et /opt/xen/domains/domU1/swap.img sont copiés dans /media/vm/domUX/ (C'est le LV créér par le script).
Et enfin les machines sont créées avec la commande xm create /etc/xen/domUX

J'ai créer vmganeti.sh pour créer les VMs dans le contexte spécifique à ganeti et vm.sh pour créer des VMs dans un contexte general.

\section {gnt-instance}

En faite ganeti gere la creation des instances, le script vmganeti.sh est obsolète.
Pour créer une instance avec ganeti il faut utiliser : 

gnt-instance add -n griffon-82.nancy.grid5000.fr -o debootstrap+default -t plain -s 1000 test1

En ayant ajouter dans le fichier /etc/hosts les instances avec une adresse IP valide par exemple :

10.0.0.1 instance1
10.0.0.2 instance2
.
.
.

J'ai donc rajouter une section dans ganeti.sh qui ajoute les instance dans /etc/hosts et qui lance leur création.

La seule OS disponible de base est : debootstrap+default


