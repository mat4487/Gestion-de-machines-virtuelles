%% LyX 1.6.7 created this file.  For more info, see http://www.lyx.org/.
%% Do not edit unless you really know what you are doing.
\documentclass[english]{article}
\usepackage[T1]{fontenc}
\usepackage[latin9]{inputenc}
\usepackage{babel}

\begin{document}

\part*{Introduction}


\section*{Machine virtuelle}


\paragraph{Une machine virtuelle est un conteneur isolé capable d\textquoteright{}exécuter
ses propres système d\textquoteright{}exploitation et applications.
Une machine virtuelle se comporte exactement comme un ordinateur physique
et contient ses propres processeur, mémoire RAM, disque dur et carte
d\textquoteright{}interface réseau virtuels.Une machine virtuelle
a pour but de générer sur une même machine un ou plusieurs environnements
d'exécution applicative. On en distingue deux types d'application
: d'une part la virtualisation par le biais d'un hyperviseur jouant
le rôle d'émulateur de système (PC ou serveur), d'autre part la virtualisation
applicative qui permet de faire tourner un application sur un poste
client quelque soit le système sous-jacent. }


\subsection{Hyperviseur }


\paragraph{La machine virtuelle avec hyperviseur est utilisée pour générer au
dessus d'un système d'exploitation serveur une couche logicielle sous
la forme d'un émulateur permettant de créer plusieurs environnements
d'exécution serveur. Cet émulateur se place comme un niveau supplémentaire
qui se greffe sur le système d'origine.}


\section*{Enjeux de la virtualisation}


\paragraph{Actuellemnt, les entreprises rencontrent des besoins qui ne sont
pas couverts.}


\paragraph{Au niveau de la sécurité, les entreprises souhaiteraient isoler les
services sur des serveurs différents. Pour la maintenance, il serait
utilser d'améliorer des services tels que la disponibilité, la migration,
la redondance,la flexibilité ou le temps de réponse. Il serait également
bienvenu de tester, déléguer l\textquoteright{}administration d\textquoteright{}un
système ...}


\paragraph{Une des solutions pour répondre à ces besoins serait d\textquoteright{}acquérir
davantage de plateformes de travail.}


\paragraph{La multiplications des serveurs pose cependant un certain nombre
de problème, augmenter sans cesse son parc informatique est impossible
pour plusieurs raison : }
\begin{itemize}
\item Tout d'abord au niveau écologie cela entrainerait un surplus de déchets
électronique,une consommation d'energie directe et de l'énergie utilisée
pour refroidir les salles serveur.
\item Au niveau de la surface utilisée, les salles machine seraient vite
encombrées, puis apparaitra des problèmes tel que la nuisance sonore,
le manque de puissance pour alimenter les salles serveur. 
\item Au niveau économique les couts d'achat, de recyclage, de fonctionnement,
de maintenance seraient trop chère. La mise en place de serveur de
virtualisation est une solution pour résoudre ces problèmes.
\end{itemize}

\paragraph{Le but de la virtualisation est de donner un environnement système
au programme pour qu\textquoteright{}il croie être dans un environnement
matériel. Pour cela, une machine virtuelle est utilisée. Ainsi, plusieurs
environnements d\textquoteright{}exécution sont créés sur une seule
machine, dont chacun émule la machine hôte. L\textquoteright{}utilisateur
pense posséder un ordinateur complet pour chaque système d\textquoteright{}exploitation
alors que toutes les machines virtuelles sont isolées entre elles.}
\end{document}
