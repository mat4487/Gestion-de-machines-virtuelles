\documentclass{report}
\usepackage[utf8]{inputenc}
\usepackage[frenchb]{babel}
%Utilisé pour la mise en forme des portions de codes%
\usepackage{listings}
%Configuration du package listings
\lstset{
language=sh,
basicstyle=\footnotesize,
numbers=left
numberstyle=\normalsize,
numbersep=7pt,
}
\begin{document}
\title{Projet Tuteurés -Outils de gestion centralisée de machines virtuelles (Tuteur : Lucas Nussbaum) }
\author{Sébastien Michaux - Augustin Bocca - Julien Tournois - Mathieu Lamouroux}
\date{Rapport hebdomadaire - 27 Janvier 2012}
\maketitle
\chapter{Travaux réalisés}
\section{Création du dépôt Git}
\paragraph{}
Mardi, nous nous sommes penchés sur la création du dépôt Git.
Ne connaissant pas ce gestionnaire de versions, il nous a fallut prendre un peu de temps pour maitriser la création du dépot sur GitHub et celle des copies de travail sur chacun des ordinateurs des membres du groupe
\paragraph{}
Après une semaine d'utilisation, nous sommes désormais plus à l'aise pour la gestion du projet via GitHub. Chaque membre à pu récupérer une copie locale du dépot et des premiers documents sont ainsi partagés.
Le projet GitHub est hébergé sur 
\begin{verbatim}
git://github.com/mat4487/Gestion-de-machines-virtuelles.git
\end{verbatim}

\section{Connexion à Grid5000}
Jeudi et Vendredi, nous nous sommes penchés sur les tutoriaux du wiki de grid5000.
Arrivés au tp2, nous avons pu tester la réservation d'une machine pour l'installation d'un debian lenny. (listing de connexion en annexe)

\chapter{À poursuivre}
\section{Git}
Le dépot Git étant fonctionnel, il dispose actuellement d'une architecture simple :
\begin{verbatim}
/Prenom_Nom -> contiennent les documents de chacun (taches réalisées, tutos,...)
/rapports -> contient les sources et les pdf de tous les (futurs) rapports
/rapports/parties_rapport -> contient les différents fichiers .tex qui constitueront les rapports intermédiaire/final
/rapports/rapports_hebdo -> contient les sources et pdf des rappots hebdomadaires ainsi qu'un squelette de rapport hebdo vide
\end{verbatim}
Cette organisation sera amenée à être évoluée au fur et à mesure de l'avancement du projet
\section{Grid5000}
Nous avons pu déployer un premier système sur un noeud de la grille de Nancy.
L'objectif va être dans un premier temps la maitrise du déploiement simultanée de plusieurs machines aptes à héberger des machines virtuelles.
\chapter{Problèmes rencontrés}
\paragraph{}
Les principaux obstacles ont étés la compréhension de Git d'une part et de Grid5000 d'autre part. N'étant pas, à l'origine, familiers de ces outils on a passé un peu de temps afin de commencer à avoir une bonne prise en main des outils.
\paragraph{}
Ces problèmes étant maintenant résolu nous n'avons pas de blocage pour l'instant.
\appendix
\chapter{Sources}
\begin{description}
\item[www.grid5000.fr : ]le wiki disponible sur le site internet de grid5000 fut notre principale source de renseignements pour le démarrage du projet.
\item[www.loria.fr/lnussbau/ : ]nous avons pu y consulter des anciens projets sur Grid5000 ce qui nous a permis d'avoir un premier aperçu de ses possibilités.
\end{description}
\chapter{Listings}
\section{Connexion à Grid5000}
Connexion à Grid5000 et déploiement d'une machine virtuelle Debian Lenny 
\begin{lstlisting}
mlamouroux@griffon-77:~/hello$ ./run_hello_mpi 
INFO: 8 cpu(s) will be used for this example
INFO: /usr/bin/mpirun will be used
Hello world from process (1) of (8) running on griffon-77.nancy.grid5000.fr
 (1) : I'm tired. I'm going to sleep a bit.
Hello world from process (2) of (8) running on griffon-77.nancy.grid5000.fr
 (2) : I'm tired. I'm going to sleep a bit.
Hello world from process (3) of (8) running on griffon-77.nancy.grid5000.fr
 (3) : I'm tired. I'm going to sleep a bit.
Hello world from process (4) of (8) running on griffon-77.nancy.grid5000.fr
 (4) : I'm tired. I'm going to sleep a bit.
Hello world from process (5) of (8) running on griffon-77.nancy.grid5000.fr
 (5) : I'm tired. I'm going to sleep a bit.
Hello world from process (6) of (8) running on griffon-77.nancy.grid5000.fr
 (6) : I'm tired. I'm going to sleep a bit.
Hello world from process (7) of (8) running on griffon-77.nancy.grid5000.fr
 (7) : I'm tired. I'm going to sleep a bit.
Hello world from process (8) of (8) running on griffon-77.nancy.grid5000.fr
 (8) : I'm tired. I'm going to sleep a bit.
 (5) : Mmmm... What? Ok, It was short but good :-)
 (7) : Mmmm... What? Ok, It was short but good :-)
 (4) : Mmmm... What? Ok, It was short but good :-)
 (1) : Mmmm... What? Ok, It was short but good :-)
 (2) : Mmmm... What? Ok, It was short but good :-)
 (3) : Mmmm... What? Ok, It was short but good :-)
 (8) : Mmmm... What? Ok, It was short but good :-)
 (6) : Mmmm... What? Ok, It was short but good :-)
mlamouroux@griffon-77:~/hello$ ls
hello_mpi  hello_mpi.c	helloworld  run_hello_mpi
mlamouroux@griffon-77:~/hello$ cd ..
mlamouroux@griffon-77:~$ ls
mlamouroux@fnancy:~$ kadeploy3 -e lenny-x64-base -m griffon-77.nancy.grid5000.fr
You do not have the rights to deploy on the node griffon-77.nancy.grid5000.fr:/dev/sda3
ERROR: You do not have the right to deploy on all the nodes
mlamouroux@fnancy:~$ echo $OAR_FILE_NODES

mlamouroux@fnancy:~$ echo $OAR_NODE_FILE

mlamouroux@fnancy:~$ oarsub -C 351753
/!\ ERROR : the job 351753 is not running. Its current state is Terminated.
mlamouroux@fnancy:~$ oarsub -l -t deploy -l'{rconsole="YES"}/nodes=1,walltime=1'
/!\ Cannot recognize the resource description : -t
mlamouroux@fnancy:~$ oarsub -l -t deploy -l '{rconsole="YES"}/nodes=1,walltime=1'
/!\ Cannot recognize the resource description : -t
mlamouroux@fnancy:~$ oarsub -I
[ADMISSION RULE] Set default walltime to 3600.
[ADMISSION RULE] Modify resource description with type constraints
Generate a job key...
OAR_JOB_ID=351759
Interactive mode : waiting...
Starting...

Connect to OAR job 351759 via the node griffon-85.nancy.grid5000.fr
mlamouroux@griffon-85:~$ logout
Connection to griffon-85.nancy.grid5000.fr closed.
Disconnected from OAR job 351759
mlamouroux@fnancy:~$ oarsub -I -t deploy -l '{rconsole="YES"}/nodes=1,walltime=3'
[ADMISSION RULE] Modify resource description with type constraints
Generate a job key...
OAR_JOB_ID=351761
Interactive mode : waiting...
Starting...

Connect to OAR job 351761 via the node fnancy.nancy.grid5000.fr
mlamouroux@fnancy:~$ cat $OAR_FILE_NODES
griffon-87.nancy.grid5000.fr
griffon-87.nancy.grid5000.fr
griffon-87.nancy.grid5000.fr
griffon-87.nancy.grid5000.fr
griffon-87.nancy.grid5000.fr
griffon-87.nancy.grid5000.fr
griffon-87.nancy.grid5000.fr
griffon-87.nancy.grid5000.fr
mlamouroux@fnancy:~$ 
mlamouroux@fnancy:~$ kadeploy3 -e lenny-x64-base -m griffon-87.nancy.grid5000.fr 
Launching a deployment ...
Performing a SetDeploymentEnvUntrusted step on the nodes: griffon-87.nancy.grid5000.fr
--- switch_pxe (griffon cluster)
  >>>  griffon-87.nancy.grid5000.fr
--- reboot (griffon cluster)
  >>>  griffon-87.nancy.grid5000.fr
  *** A soft reboot will be performed on the nodes griffon-87.nancy.grid5000.fr
--- wait_reboot (griffon cluster)
  >>>  griffon-87.nancy.grid5000.fr

--- send_key_in_deploy_env (griffon cluster)
  >>>  griffon-87.nancy.grid5000.fr
  *** No key has been specified
--- create_partition_table (griffon cluster)
  >>>  griffon-87.nancy.grid5000.fr
--- format_deploy_part (griffon cluster)
  >>>  griffon-87.nancy.grid5000.fr
--- mount_deploy_part (griffon cluster)
  >>>  griffon-87.nancy.grid5000.fr
--- format_tmp_part (griffon cluster)
  >>>  griffon-87.nancy.grid5000.fr
  *** Bypass the format of the tmp part
--- format_swap_part (griffon cluster)
  >>>  griffon-87.nancy.grid5000.fr
  *** Bypass the format of the swap part
Performing a BroadcastEnvKastafior step on the nodes: griffon-87.nancy.grid5000.fr
--- send_environment (griffon cluster)
  >>>  griffon-87.nancy.grid5000.fr
  *** Broadcast time: 46 seconds
--- manage_admin_post_install (griffon cluster)
  >>>  griffon-87.nancy.grid5000.fr
--- manage_user_post_install (griffon cluster)
  >>>  griffon-87.nancy.grid5000.fr
--- send_key (griffon cluster)
  >>>  griffon-87.nancy.grid5000.fr
--- install_bootloader (griffon cluster)
  >>>  griffon-87.nancy.grid5000.fr
Performing a BootNewEnvClassical step on the nodes: griffon-87.nancy.grid5000.fr
--- switch_pxe (griffon cluster)
  >>>  griffon-87.nancy.grid5000.fr
--- umount_deploy_part (griffon cluster)
  >>>  griffon-87.nancy.grid5000.fr
--- reboot_from_deploy_env (griffon cluster)
  >>>  griffon-87.nancy.grid5000.fr
--- set_vlan (griffon cluster)
  >>>  griffon-87.nancy.grid5000.fr
  *** Bypass the VLAN setting
--- wait_reboot (griffon cluster)
  >>>  griffon-87.nancy.grid5000.fr
Nodes correctly deployed on cluster griffon
griffon-87.nancy.grid5000.fr
mlamouroux@fnancy:~$ 
mlamouroux@fnancy:~$ ssh root@griffon-87.nancy.grid5000.fr
Warning: Permanently added 'griffon-87.nancy.grid5000.fr,172.16.65.87' (RSA) to the list of known hosts.
root@griffon-87.nancy.grid5000.fr's password: 
Linux griffon-87.nancy.grid5000.fr 2.6.26-2-amd64 #1 SMP Wed Sep 21 03:36:44 UTC 2011 x86_64

Lenny-x64-base-2.4 (Image based on Debian Lenny for AMD64/EM64T)
  Maintained by support-staff <support-staff@lists.grid5000.fr>

Applications
  * Text: Vim, nano
  * Script: Perl, Python, Ruby
  (Type "dpkg -l" to see complete installed package list)

Misc
  * SSH has X11 forwarding enabled
  * Max open files: 65536

More details: https://www.grid5000.fr/mediawiki/index.php/Lenny-x64-base-2.4
griffon-87:~# who
root     pts/0        2012-01-27 14:15 (fnancy.nancy.grid5000.fr)
griffon-87:~# 

\end{lstlisting}
\cleardoublepage

\section{TP2 - Cluster Experiment}
TP2 : Cluster Experiment sur le Wiki de Grid5000
\begin{lstlisting}
lien : http://www.grid5000.fr/mediawiki/index.php/Cluster_experiment

configuration du proxy:
étape1: liste les proxy configuré en http et https
	commande: echo http_proxy=$http_proxy ; echo https_proxy=$https_proxy
	résultat:
		http_proxy=
		https_proxy=

étape2: Initialisé le proxy
	commande: export http_proxy="http://proxy:3128" ; export https_proxy="http://proxy:3128"
	résultat:
		http_proxy=http://proxy:3128
		https_proxy=http://proxy:3128

récupération de la tarball hello
	 commande: wget --no-check-certificate https://gforge.inria.fr/frs/download.php/26756/hello.tgz
décompression de la tarballe
	 commande: tar -xvzf ~/hello.tgz -C ~/
___________________________________________________________________________
commande 	       	     résultat					   |
oarstat			     permet de voir toutes les sumissions de job   |
									   |
résultat de la commande:						   |
ob id     Name           User           Submission Date     S Queue	   |
---------- -------------- -------------- ------------------- - ----------  |
351629                    malexand       2012-01-26 08:15:14 R default     |
351635                    tbuchert       2012-01-26 09:45:47 R default     |
351637                    trakotoarivelo 2012-01-26 09:54:11 R default     |
351639                    lsarzyniec     2012-01-26 10:23:05 R default     |
351640                    ejeanvoine     2012-01-26 10:35:00 R default     |
351647                    mquinson       2012-01-26 11:03:16 R default     |
351464                    malexand       2012-01-24 08:34:10 W default     |
351632                    gbrand         2012-01-26 09:40:15 W default     |
351633                    gbrand         2012-01-26 09:40:18 W default     |
351644                    falvaresdeoliv 2012-01-26 10:47:15 W default     |
351645                    falvaresdeoliv 2012-01-26 10:47:20 W default     |
351650     Gridmix        pcosta         2012-01-26 12:02:23 W default     |
351651                    jmontanier     2012-01-26 14:18:20 W default     |
----------------------------------------------------------------------------
oarstat -f			liste avec détail chaque réservations
----------------------------------------------------------------------------
oarstat -f -j OAR_JOB_ID	liste le détail d'une réservation avec un job id
----------------------------------------------------------------------------
oarstat -s -j OAR_JOB_ID	montre le status d'un job spécific
----------------------------------------------------------------------------
oarstat -u LOGIN		montre les réservation d'un utilisateur
----------------------------------------------------------------------------
oarnodes			liste les propriétés du cluster
----------------------------------------------------------------------------
oarprint host -P host,cpu,core -F "host: % cpu: % core: %" -C+			
liste les propriété du noeud utilisé(a utilisé lors de l(utilisation d'un noeud)
----------------------------------------------------------------------------
oarstat -j OAR_JOB_ID -p | oarprint core -P host,cpuset,memcore -F "%[%] (%)" -f - | sort
idem mais peut ce lancer dans le fronted
----------------------------------------------------------------------------
oarsub -I			permet de réserver un noeud pour 1H

résultat:
jtournois@fnancy:~$ oarsub -I
[ADMISSION RULE] Set default walltime to 3600.
[ADMISSION RULE] Modify resource description with type constraints
Generate a job key...
OAR_JOB_ID=351654
Interactive mode : waiting...
Starting...
----------------------------------------------------------------------------
env | grep -i ^oar		liste les variables d'environement
----------------------------------------------------------------------------
cat $OAR_NODE_FILE		liste le noeud utilisé
----------------------------------------------------------------------------

étape3: lancement du script hello
	commande: ./run_hello.mpi
résultat:
Hello world from process (1) of (8) running on griffon-91.nancy.grid5000.fr
 (1) : I'm tired. I'm going to sleep a bit.
Hello world from process (2) of (8) running on griffon-91.nancy.grid5000.fr
 (2) : I'm tired. I'm going to sleep a bit.
Hello world from process (5) of (8) running on griffon-91.nancy.grid5000.fr
 (5) : I'm tired. I'm going to sleep a bit.
Hello world from process (6) of (8) running on griffon-91.nancy.grid5000.fr
 (6) : I'm tired. I'm going to sleep a bit.
Hello world from process (7) of (8) running on griffon-91.nancy.grid5000.fr
 (7) : I'm tired. I'm going to sleep a bit.
Hello world from process (3) of (8) running on griffon-91.nancy.grid5000.fr
 (3) : I'm tired. I'm going to sleep a bit.
Hello world from process (4) of (8) running on griffon-91.nancy.grid5000.fr
 (4) : I'm tired. I'm going to sleep a bit.
Hello world from process (8) of (8) running on griffon-91.nancy.grid5000.fr
 (8) : I'm tired. I'm going to sleep a bit.
 (5) : Mmmm... What? Ok, It was short but good :-)
 (6) : Mmmm... What? Ok, It was short but good :-)
 (7) : Mmmm... What? Ok, It was short but good :-)
 (2) : Mmmm... What? Ok, It was short but good :-)
 (1) : Mmmm... What? Ok, It was short but good :-)
 (8) : Mmmm... What? Ok, It was short but good :-)
 (4) : Mmmm... What? Ok, It was short but good :-)
 (3) : Mmmm... What? Ok, It was short but good :-)

test les processeur disponibles
----------------------------------------------------------------------------
 Ctrl-D or exit			exit du noeud
----------------------------------------------------------------------------
oarsub ~/hello/run_hello_mpi -O ~/hello_mpi.log
redirige la sortie dans un fichier
----------------------------------------------------------------------------
oarsub -C OAR_JOB_ID		connexion a un job existant
----------------------------------------------------------------------------
oarsub -I -l nodes=2		réservation de deux noeud
----------------------------------------------------------------------------
oarsh OTHER_NODE_HOSTNAME	connexion a un autre noeud
----------------------------------------------------------------------------
oarsh OTHER_NODE_HOSTNAME ps -C hello_mpi
permet de lancer le script sur un noeud distant
----------------------------------------------------------------------------
oarsub -I -t container -l nodes=4,walltime=0:45:00
réservation de 4 noeud pour 45 min
----------------------------------------------------------------------------
oarsub -I -t inner=containerJobID -l nodes=3,walltime=0:15:00
permet de rajuster la réservation précédente a 3 noeud pour 15min
----------------------------------------------------------------------------
oardel OAR_JOB_ID		supprime un job id
----------------------------------------------------------------------------
\end{lstlisting}

\end{document}
