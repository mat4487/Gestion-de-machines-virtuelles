\documentclass{report}
\usepackage[utf8]{inputenc}
\usepackage[frenchb]{babel}
\begin{document}
\title{Projet Tuteurés -Outils de gestion centralisée de machines virtuelles (Tuteur : Lucas Nussbaum) }
\author{Sébastien Michaux - Augustin Bocca - Julien Tournois - Mathieu Lamouroux}
\date{27 Janvier 2012}
\maketitle
\chapter{Travaux réalisés}
\section{Mise en place de git}
Le dépot a été crée sous l'adresse 
\begin{verbatim}
git://github.com/mat4487/Gestion-de-machines-virtuelles.git
\end{verbatim}
\section{Utilisation de Grid5000}
\subsection{Connexion}
Après avoir déchiffré le wiki anglais, nous avons réussi à nous connecter sur le frontend de nancy.
\subsection{Réservation}
\subsection{Installation d'une machine}
\chapter{À poursuivre}

\chapter{Problèmes rencontrés}
\section{La connexion à Grid5000}
Grid5000 s'utilise via une connexion SSH. Mais il est nécessaire de se connecter à un serveur avant d'initier une nouvelle connexion depuis ce point vers le \emph{frontend} du site de Nancy.
\section{Utilisation de git}

\appendix
\chapter{Sources}
\begin{description}
\item[www.grid5000.fr : ]le wiki disponible sur le site internet de grid5000 fut principale source de renseignements pour le démarrage du projet.
\item["www.loria.fr/lnussbau/ : "]nous avons pu y consulter des anciens projets sur Grid5000 ce qui nous a permis d'avoir un premier aperçu de ses possibilités.
\end{description}

\end{document}
