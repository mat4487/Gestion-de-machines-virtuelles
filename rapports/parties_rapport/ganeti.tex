\lstset{language=bash}
\chapter{installation de ganeti}
\section{installer ganeti}
installer ganeti :
\begin{lstlisting}
apt-get install ganeti2 ganeti-htools ganeti-instance-debootstrap
\end{lstlisting}
Dans /etc/host ajouter l'adresse IP du node avec le nom de l'host ainsi que l'adresse du cluster :

10.144.64.1 cluster1
172.16.65.56 griffon-56.nancy.grid5000.fr

Lorsque on aura plusieur nodes il faudra une autre adresse Ip au cluster car elle doit etre accessible � tous les nodes

Dans /boot/ creer des liens symboliques :
\begin{lstlisting}
ln -s vmlinuz-2.6.32-5-xen-amd64 vmlinuz-2.6.xenU
ln -s initrd.img-2.6.32-5-xen-amd64 initrd.img-2.6.xenU
\end{lstlisting}

Creer un LVM d'au moins 20Go (Obligatoire) :
\begin{lstlisting}
umout /dev/sda5
pvcreate /dev/sda5
vgcreate xenvg /dev/sda5
\end{lstlisting}
Dans /ect/network/interface remplacer le paragraphe de eth0 par celle du brige xen-br0 :
\begin{lstlisting}
auto xen-br0
iface xen-br0 inet static
 address 172.16.65.56 #Adresse du node
 netmask 255.255.240.0 
 network 172.16.64.0
 broadcast 172.16.79.255
 gateway 10.144.64.254
 bridge_ports eth0
 bridge_stp off
 bridge_fd 0
 \end{lstlisting}
\Chatper{Utilisation de ganeti}

Initialiser le cluster avec ganeti :
\begin{lstlisting}
gnt-custer init --no-drbd-storage cluster1
\end{lstlisting}
ajouter le node
gnt-node add griffon-56.nancy.grid5000.fr

et verifier avec :
gnt-node list

Qui devrait renvoyer quelque chose comme cela :

Node                         DTotal  DFree MTotal MNode MFree Pinst Sinst
griffon-56.nancy.grid5000.fr 283.2G 283.2G      -     -     -     0     0


