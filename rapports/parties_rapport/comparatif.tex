\chapter{Comparatif}
\newcolumntype{M}[1]{>{\raggedright}m{#1}}
Voici un petit tableau de comparaisons des différentes solutions de gestion centralisée, en fonction de ce que nous avons pu tester.
%\begin{table}
\begin{center}
\begin{tabular}{|c|M{2.7cm}|M{2.7cm}|M{2.7cm}|M{2.7cm}|}
\hline
 & OpenXenManager & Ganeti & Virt-Manager & Archipel \tabularnewline
\hline
Documentation & Quasi inexistante & Bonne : Toutes les informations se trouvent assez facilement sur le site de ganeti. & Quelques sites proposent des tutoriaux. Nous pouvons qualifier la documentation de bonne bien que les configuration poussées nécessitent des recherches personnelles & Détaillée : Les tutoriels sont assez complets et la documentation sur le logiciel reste relativement jeune et donc complète sur le site du projet archipel \tabularnewline
\hline
Communauté & Très peu de sites traitent du sujet. Nous avons trouvé très peu de réponse à nos problèmes & Présente : La communauté reste encore active même si le projet est abouti. & On peut trouver facilement des sites traitant du sujet mais encore une fois, seules les questions de bases sont évoquées. Nos besoins nous poussant souvent à rechercher nous-mêmes une solution.& Non-testé : Nous n’avons pas pu mettre en œuvre ce logiciel.\tabularnewline
\hline
Maturité & Projet toujours en cours de développement & Aboutie : produit complet beaucoup de fonctionnalisées disponibles. & Projet mature mais disposant de trop peu de fonctionnalités pour permettre une pleine exploitation des possibilitées de Grid5000 & Jeune : Le projet est actuellement en beta 5. \tabularnewline
\hline
\end{tabular}
\end{center}
\newpage
\begin{center}
\begin{tabular}{|c|M{2.7cm}|M{2.7cm}|M{2.7cm}|M{2.7cm}|}
\hline
 & OpenXenManager & Ganeti & Virt-Manager & Archipel \tabularnewline
\hline
Installation & Non-testé : Nous n’avons pas pu mettre en œuvre ce logiciel. & Problématique : installation facile mais la configuration peut être complexe celons l'architecture du réseau et les besoins des utilisateurs & Installation aisée via le logiciel de gestion de paquets apt-get& Longue : L’installation reste complexe car beaucoup de choses restent à configurer pour que le logiciel fonctionne sur la plateforme.\tabularnewline
\hline
Réseau & Non-testé : Nous n’avons pas pu mettre en œuvre ce logiciel. & Passable : Déploiement plutôt facile tant sur un réseau complexe que sur un réseau simple, la gestion des ip des machines virtuelles reste compliquée & Configuration automatique, virt s'occupe de configurer le bridge & Non-testé : Nous n’avons pas pu mettre en œuvre ce logiciel.\tabularnewline
\hline
Sécurité & Non-testé : Nous n’avons pas pu mettre en œuvre ce logiciel. & Bonne : Le dialogue entre tous les nœuds est effectué en ssh avec transmission de la clé du maitre sur les esclaves. & Communications entres les noeuds au-dessus d'un tunnel SSH & Très bonne : Le logiciel ce base sur le protocole XMPP (SASL et TLS) pour discuter avec les nœuds.\tabularnewline
\hline
Simplicité & Bonne : possède une interface graphique. Mais nous n'avons pas pu mettre en oeuvre ce logiciel donc nous n'avons pas d'avis très critique. & Mitigé : Pour une architecture classique cela reste assez facile à mettre en œuvre. & Bonne : Interface graphique assez intuitive & Non-testé : Nous n’avons pas pu mettre en œuvre ce logiciel. \tabularnewline
\hline
Flexibilité & Non-testé : Nous n’avons pas pu mettre en œuvre ce logiciel. & Très bonne : Tant que l'on reste dans une architecture classique, sa grande modularité est un atout qui permet de facilement le déployer sur une architecture multi-sites. & Assez faible, on ne peut pas rajouter de fonctionnalitées& Non-testé : Nous n’avons pas pu mettre en œuvre ce logiciel. \tabularnewline
\hline
\end{tabular}
\end{center}
%\end{table}
