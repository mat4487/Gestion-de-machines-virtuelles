\lstset{language=bash}
\chapter{OpenXENManager}
XenseMaking Project d�veloppe un client lourd, ainsi qu'un client
web, pour manager XenServer. C'est un clone du XenCenter, qui fonctionne
avec Linux, BSD, Windows et MacOSX, alors que le XenCenter ne fonctionne
qu'avec Windows. OpenXenManager/OpenXenCenter un le client lourd qui
permet de manager XenServer.Il a �t� d�velopp� en Python avec pygtk
et gtk-vnc.
Les fonctionnalit�s actuellement impl�ment�es sont les suivantes :
\begin{itemize}
\item - monitoring des machines virtuelles - acc�s � la console des machines
virtuelles
\item - op�rations d'administration (d�marrage, arr�t, reboot, ...)
\item - cr�ation de machines virtuelles
\end{itemize}
\section{Guide d'installation d'OpenXenManager}
Pour l'installation nous avons besoin des paquets suivant:
\begin{lstlisting}
apt-get install subversion bzip2 python-glade2 python-gtk-vnc shared-mime-info graphviz
\end{lstlisting}
On t�l�charge la derni�re version d'openxenmanager dans le d�p�t
subversion
\begin{lstlisting}
svn co https://openxenmanager.svn.sourceforge.net/svnroot/openxenmanager openxenmanager
\end{lstlisting}
On se d�place dans le r�pertoire trunk:
\begin{lstlisting}
cd openxenmanager/trunk
\end{lstlisting}
Finalement on lance openxenmanager avec la commande suivante
\begin{lstlisting}
python window.py
\end{lstlisting}
Une interface graphique d'openxenmanger apparait.
