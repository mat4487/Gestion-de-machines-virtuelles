\section{Listing d'unne installation de Ganetti}
\begin{lstlisting}
#!/bin/bash
#passage en wheezy
rm /etc/apt/sources.list
echo "## wheezy security" > /etc/apt/sources.list
echo "deb http://security.debian.org/ wheezy/updates main contrib non-free" >> /etc/apt/sources.list
echo "deb-src http://security.debian.org/ wheezy/updates main contrib non-free" >> /etc/apt/sources.list
echo " " >> /etc/apt/sources.list
echo "#wheezy" >> /etc/apt/sources.list
echo "deb http://ftp.fr.debian.org/debian/ wheezy main contrib non-free" >> /etc/apt/sources.list
echo "deb-src http://ftp.fr.debian.org/debian/ wheezy main contrib non-free" >> /etc/apt/sources.list
#Installation de ganeti 
apt-get update
apt-get dist-upgrade -y --force-yes
apt-get install -y --force-yes ganeti2 ganeti-htools ganeti-instance-debootstrap
echo "Ajout du node dans /etc/hosts"
hostname=`cat /etc/hostname`
#recuperation des variables
#ip du node
ifconfig eth0 > troll
ipnode=`head -2 troll | tail -1 | cut -d':' -f2 | cut -d' ' -f1`
echo \$ipnode \$hostname >> /etc/hosts
#ip du broadcast
ipbroadcast=`head -2 troll | tail -1 | cut -d'B' -f2 | cut -d':' -f2 | cut -d' ' -f1`
#ip du masque de sous reseau
ipmask=`head -2 troll | tail -1 | cut -d'M' -f2 | cut -d':' -f2 | cut -d' ' -f1`
#ip du reseau
ipnetwork=`head -1 ipnetwork`
#ip de la passerelle
ipgateway=`head -1 ipgateway`
#ajout de cluster1 dans dans /etc/hosts
echo "ajout de cluster1 dans /etc/hosts"
ipcluster=`cat ipcluster`
echo \$ipcluster "cluster1" >> /etc/hosts
#Dans /boot/ creer des liens symboliques :
ln -s /boot/vmlinuz-2.6.32-5-xen-amd64 /boot/vmlinuz-2.6.xenU
ln -s /boot/initrd.img-2.6.32-5-xen-amd64 /boot/initrd.img-2.6.xenU
#Pour le moment changera surement.
echo "creation du LVM"
umount /dev/sda5
pvcreate /dev/sda5
vgcreate xenvg /dev/sda5
#Creation du bridge xen-br0
echo " " >> /etc/network/interfaces
echo "auto xen-br0"  >> /etc/network/interfaces
echo "iface xen-br0 inet static" >> /etc/network/interfaces 
echo "address" \$ipnode >> /etc/network/interfaces
echo "netmask " \$ipmask >> /etc/network/interfaces
echo "network" \$ipnetwork >> /etc/network/interfaces
echo "gateway"  \$ipgateway >> /etc/network/interfaces
echo "broadcast" \$ipbroadcast>> /etc/network/interfaces
echo "bridge_ports eth0" >> /etc/network/interfaces
echo "bridge_stp off" >> /etc/network/interfaces
echo "bridge_fd 0" >> /etc/network/interfaces
#Suppression des ligne de eth0
sed -i '9d' /etc/network/interfaces
sed -i '9d' /etc/network/interfaces
#supression des fichier temporaires
rm troll ipcluster  ipgateway  ipnetwork
#initialisation du cluster
gnt-cluster init --no-drbd-storage cluster1
#ajouter le node 
gnt-node add \$hostname
#et verifier
gnt-node list
\end{lstlisting}

