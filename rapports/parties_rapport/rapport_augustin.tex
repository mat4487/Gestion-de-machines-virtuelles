\Chapter{installation de ganeti}
\section{installer ganeti}


installer ganeti :

apt-get install ganeti2 ganeti-htools ganeti-instance-debootstrap

Dans /etc/host ajouter l'adresse IP du node avec le nom de l'host ainsi que l'adresse du cluster :

10.144.64.1 cluster1
172.16.65.56 griffon-56.nancy.grid5000.fr

Lorsque on aura plusieur nodes il faudra une autre adresse Ip au cluster car elle doit etre accessible à tous les nodes

Dans /boot/ creer des liens symboliques :

ln -s vmlinuz-2.6.32-5-xen-amd64 vmlinuz-2.6.xenU
ln -s initrd.img-2.6.32-5-xen-amd64 initrd.img-2.6.xenU


Creer un LVM d'au moins 20Go (Obligatoire) :

umout /dev/sda5
pvcreate /dev/sda5
vgcreate xenvg /dev/sda5

Dans /ect/network/interface remplacer le paragraphe de eth0 par celle du brige xen-br0 :

auto xen-br0
iface xen-br0 inet static
 address 172.16.65.56 #Adresse du node
 netmask 255.255.240.0 
 network 172.16.64.0
 broadcast 172.16.79.255
 gateway 10.144.64.254
 bridge_ports eth0
 bridge_stp off
 bridge_fd 0
 
\Chatper{Utilisation de ganeti}

Initialiser le cluster avec ganeti :
gnt-custer init --no-drbd-storage cluster1

ajouter le node
gnt-node add griffon-56.nancy.grid5000.fr

et verifier avec :
gnt-node list

Qui devrait renvoyer quelque chose comme cela :

Node                         DTotal  DFree MTotal MNode MFree Pinst Sinst
griffon-56.nancy.grid5000.fr 283.2G 283.2G      -     -     -     0     0


