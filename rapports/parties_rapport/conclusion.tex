\chapter{Conclusion}
\emph{L'objectif de ce projet était de déployer et d'évaluer les différentes solutions libres permettant de gérer la gestion de machines virtuelles sur la plateforme Grid 5000.\\
\\
Cet objectif est partiellement atteint. En effet, nous avons pu tester que deux solutions de gestion : Ganeti et virt-manager.\\
\\
Ce projet étant ambitieux, nous nous sommes vite heurtés à de nombreux problèmes, que ce soit dû aux solutions de gestion ou à leur intégration sur la plateforme Grid 5000, notamment en ce qui concerne le réseau pour ganeti ou les interfaces graphiques pour les autres solutions.\\
<<<<<<< HEAD
Tous ces problèmes nous ont montré la complexité d'utiliser une telle plate-forme, et leurs résolutions nous a souvent retardé mais nous a appris à chercher des solutions afin de contourner les différents problèmes.\\
=======
\\
Tous ces problèmes nous ont montré la complexité d'utiliser une telle plate-forme, et leurs résolutions nous a souvent retardé mais nous ont appris à chercher des solutions afin de contourner les différents problèmes.\\
\\
>>>>>>> fb93e291bfb4b7b9a4d9cf2c64ddc3421c479ad8
Ce projet nous a également apporté une expérience de travail en équipe sur un projet de longue durée, en le découpant en tâches et en répartissant le travail.\\
\\
Ce projet a été pour nous une chance de découvrir un environnement aussi complexe que Grid 5000 qui est très intéressant, ce qui nous a permis d'acquérir de l'expérience en administration systèmes et réseaux et d'approfondir nos connaissances dans le domaine de la virtualisation et de la gestion de machines virtuelles.}
