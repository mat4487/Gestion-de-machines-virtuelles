\documentclass{report}[a4paper,12pt]
\usepackage[utf8]{inputenc}
\usepackage[frenchb]{babel}
\usepackage[T1]{fontenc}
\usepackage{graphicx}
\usepackage{array}
%\usepackage{fancyhdr}
\usepackage{listings}
%\pagestyle{fancy}
\usepackage{color}
%\definecolor{javared}{rgb}{0.6,0,0} % for strings
%\definecolor{javagreen}{rgb}{0.25,0.5,0.35} % comments
%\definecolor{javapurple}{rgb}{0.5,0,0.35} % keywords
%\definecolor{javadocblue}{rgb}{0.25,0.35,0.75} % javadoc
\definecolor{hellgelb}{rgb}{1,1,0.8}
\definecolor{colKeys}{rgb}{0,0,1}
\definecolor{colIdentifier}{rgb}{0,0,0}
\definecolor{colComments}{rgb}{1,0,0}
\definecolor{colString}{rgb}{0,0.5,0}
\lstset{%
language=ruby%
morekeywords={}%
float=hbp,%
basicstyle=\ttfamily\small, %
identifierstyle=\color{colIdentifier}, %
keywordstyle=\color{colKeys}, %
stringstyle=\color{colString}, %
commentstyle=\color{colComments}, %
columns=flexible, %
tabsize=2, %
frame=single, %
extendedchars=true, %
showspaces=false, %
showstringspaces=false, %
numbers=left, %
numberstyle=\tiny, %
breaklines=true, %
backgroundcolor=\color{hellgelb}, %
breakautoindent=true, %
captionpos=b%
}
\title{Projet Tuteurés -Outils de gestion centralisée de machines virtuelles\\\\Tuteur : Lucas Nussbaum}
\author{Sébastien Michaux - Augustin Bocca - Julien Tournois - Mathieu Lamouroux}
\date{\today}
\begin{document}
\maketitle
\begin{abstract}
%%%%%%%%%%%%%%%%%%%%%%
%% Résumé du projet %%
%%%%%%%%%%%%%%%%%%%%%%
\end{abstract}
Remerciements


Toujours en cours de réflexion :)

\newpage
\tableofcontents
\newpage
\chapter{Introduction}
\section{Présentation du projet}
%\paragraph{Objectif}
Mettre en place, évaluer et comparer différents outils permettant de gérer de manière
centralisée et automatisée une infrastructure basée sur des machines virtuelles: Ganeti,
OpenXenManager, virt-manager, Archipel...
\section{Introduction à la virtualisation}
\subsection{Machine virtuelle}


Une machine virtuelle est un conteneur isolé capable d exécuter
ses propres système d exploitation et applications.
Une machine virtuelle se comporte exactement comme un ordinateur physique
et contient ses propres processeur, mémoire RAM, disque dur et carte
d interface réseau virtuels.Une machine virtuelle
a pour but de générer sur une même machine un ou plusieurs environnements
d'exécution applicative. On en distingue deux types d'application
: d'une part la virtualisation par le biais d'un hyperviseur jouant
le rôle d'émulateur de système (PC ou serveur), d'autre part la virtualisation
applicative qui permet de faire tourner un application sur un poste
client quelque soit le système sous-jacent.

\subsection{Hyperviseur }


La machine virtuelle avec hyperviseur est utilisée pour générer au
dessus d'un système d'exploitation serveur une couche logicielle sous
la forme d'un émulateur permettant de créer plusieurs environnements
d'exécution serveur. Cet émulateur se place comme un niveau supplémentaire qui se greffe sur le système d'origine.
\newpage
\subsection{Enjeux de la virtualisation}


Actuellemnt, les entreprises rencontrent des besoins qui ne sont
pas couverts.

Au niveau de la sécurité, les entreprises souhaiteraient isoler les
services sur des serveurs différents. Pour la maintenance, il serait
utilser d'améliorer des services tels que la disponibilité, la migration,la redondance,la flexibilité ou le temps de réponse. Il serait également bienvenu de tester, déléguer l administration d un système ...


Une des solutions pour répondre à ces besoins serait d acquérir davantage de plateformes de travail.


La multiplications des serveurs pose cependant un certain nombre
de problème, augmenter sans cesse son parc informatique est impossible
pour plusieurs raison : 
\begin{itemize}
\item Tout d'abord au niveau écologie cela entrainerait un surplus de déchets électronique,une consommation d'energie directe et de l'énergie utilisée pour refroidir les salles serveur.
\item Au niveau de la surface utilisée, les salles machine seraient vite encombrées, puis apparaitra des problèmes tel que la nuisance sonore, le manque de puissance pour alimenter les salles serveur. 
\item Au niveau économique les couts d'achat, de recyclage, de fonctionnement, de maintenance seraient trop chère. La mise en place de serveur de virtualisation est une solution pour résoudre ces problèmes.
\end{itemize}

Le but de la virtualisation est de donner un environnement système
au programme pour qu il croie être dans un environnement
matériel. Pour cela, une machine virtuelle est utilisée. Ainsi, plusieurs
environnements d exécution sont créés sur une seule
machine, dont chacun émule la machine hôte. L utilisateur
pense posséder un ordinateur complet pour chaque système d exploitation
alors que toutes les machines virtuelles sont isolées entre elles.

\subsection{Histoire de la virtualisation}

La virtualisation est un concept qui a été mis au point pour la première
fois dans les années 1960 pour permettre la partition dune vaste gamme de matériel mainframe et optimiser l'utilisation du matériel. De nos jours, les ordinateurs basés sur l'architecture x86 sont confrontés aux mêmes problèmes de rigidité et de sous-utilisation que les mainframes dans les années 1960. VMware a inventé la virtualisation pour la plate-forme x86 dans les années 1990 afin de répondre notamment aux problèmes de sous-utilisation, et a surmonté les nombreux défis émergeant au cours de ce processus. Aujourd'hui, VMware est devenu le leader mondial de la virtualisation x86 avec plus de 190 000 clients, dont la totalité des membres du classement Fortune 100.

\subsection{Au commencement, la virtualisation des mainframes}

La virtualisation a été mise en œuvre pour la première fois il y
a plus de 30 ans par IBM pour partitionner logiquement des mainframes
en machines virtuelles distinctes. Ces partitions permettaient un
traitement « multitâche », à savoir l exécution simultanée de plusieurs applications et processus. Étant donné que les mainframes consommaient beaucoup de ressources en même temps, le partitionnement
constituait un moyen naturel de tirer pleinement parti de l investissement matériel.
\newpage
\section{Présentation de Grid5000}
\begin{figure}
\begin{center}
\includegraphics{images/logo.png}
\caption{Logo de Grid5000}
\end{center}

\end{figure}
Aujourd’hui, grâce à Internet, il est possible
d’interconnecter des machines du monde entier pour
traiter et stocker des masses de données. Cette collection
hétérogène et distribuée de ressources de stockage et de
calcul a donné naissance à un nouveau concept : les
grilles informatiques.

L’idée de mutualiser les ressources
informatiques vient de plusieurs facteurs, évolution de la
recherche en parallélisme qui, après avoir étudié les
machines homogènes, s’est attaquée aux environnements
hétérogènes puis distribués ; besoins croissants des
applications qui nécessitent l’utilisation toujours plus
importante de moyens informatiques forcément répartis.

La notion de grille peut avoir plusieurs sens suivant le
contexte : grappes de grappes, environnements de type
GridRPC (appel de procédure à distance sur une grille).,
réseaux pair-à-pair, systèmes de calcul sur Internet, etc...
Il s’agit d’une manière générale de systèmes dynamiques,
hétérogènes et distribués à large échelle. Un grand
nombre de problématiques de recherche sont soulevées
par les grilles informatiques. Elles touchent plusieurs
domaines de l’informatique :algorithmique,
programmation, intergiciels, applications, réseaux.

L’objectif de GRID’5000 est de construire un instrument
pour réaliser des expériences en informatique dans le
domaine des systèmes distribués à grande échelle (GRID).

Cette plate-forme, ouverte depuis 2006 aux chercheurs de
la communauté grille, regroupe un certain nombre de sites
répartis sur le territoire national. Chaque site héberge une
ou plusieurs grappes de processeurs. Ces grappes sont
alors interconnectées via une infrastructure réseau dédiée
à 10 Gb/s fournie par RENATER. À ce jour, GRID’5000
est composé de 9 sites: Lille, Rennes, Orsay, Nancy,
Bordeaux, Lyon, Grenoble, Toulouse et Nice.

Début 2007, GRID’5000 regroupait plus de 2500 processeurs et près
de 3500 cœurs.

\newpage
\begin{figure}
\begin{center}
\includegraphics{images/g5k.png}
\\
\underline{\textit{Répartition des sites}}
\end{center}
\end{figure}



  \subsection{Infrastructure des sites}
Chaque site héberge :
\begin{itemize}
\item un frontend, serveur permettant d'accéder aux clusters disponibles ,
\item un serveur de données, pour centraliser les données utilisateurs ,
\item plusieurs clusters, c'est-à-dire des grappes de machines homogènes, appelées noeuds (nodes).
\end{itemize}
\begin{center}
\includegraphics[width=10cm,height=15cm]{images/g5k1.png}
\\
\underline{\textit{Architecture Grid5000}}
%%\includegraphics{images/g5k1.png}
\end{center}
L'utilisateur de Grid 5000 accède à chaque site par son frontend en utilisant le protocole SSH.\\
Commande:
\begin{lstlisting}
ssh jdoe@access.grid5000.fr
\end{lstlisting}
Sur tous les serveurs du site, un répertoire home, local à chaque site, est monté avec NFS 2 .
A partir du frontend, il est possible d'accéder aux machines des clusters en exectuant des réservations à l'aide de la commande:
\begin{lstlisting}
oarsub
\end{lstlisting}
Gràce à notre tuteur, M. Lucas Nussbaum nous avons pu visiter la salle serveurs du site de Nancy située au Loria, 
ainsi qu'une présentation de la plate-forme (matériel utilisé, connexions réseau,
administration).
\quotation\textit{Une description détaillée du site de Nancy est disponible sur le site de Grid 5000.}

  \subsection{Réseau}
Les sites et toutes les machines qu'ils comprennent sont interconnectés par RENATER 3 en 10Gbits/s. De
plus, chaque site peut disposer de plusieurs réseaux locaux 4 :
\begin{itemize}
\item réseau en ethernet, 1 Gb/s
\item réseaux hautes performances (Infiniband 20 Gb/s ou 10 Gb/s, et Myrinet 20 Gb/s)
\end{itemize}

  \subsection{Environnement logiciel}
Tous les serveurs de Grid 5000 fonctionnent sous Debian GNU/Linux.
A partir du frontend, l'utilisateur peut réserver des machines en utilisant la suite de logiciels OAR dédiée à
la gestion de ressources de clusters, et déployer ses propres images de systèmes à l'aide des outils kadeploy.
Il y a deux types de réservation :
\begin{itemize}
\item par défaut, pour des besoins de calcul avec OpenMPI ;
\item pour le déploiement d'environnements (deploy ).
\end{itemize}


\section{Répartition des tâches}
Nous avons commencé par prendre en main Grid5000 durant les 2 premières semaines du projet. Pour ce faire nous avons suivi avec soin les tutoriels mis à notre disposition sur le site www.grid5000.fr.

Une fois les manipulations de bases bien assimilées. Nous nous sommes divisés en 2 sous-groupes pour tester les différents outils du projet :
\begin{itemize}
  \item Julien et Augustin se sont chargés de Ganeti.
  \item Sébatien et Mathieu pour OpenXenManager.
\end{itemize}
A l'heure actuelle nous n'avons pas encore eu le temps de nous projeter sur les autres outils et ne les avons donc pas répartis.


%\documentclass{beamer}

%\usepackage[utf8x]{inputenc}
%\usepackage{default}
%\usepackage{graphicx}

%\begin{document}

\begin{frame}
  \begin{center}
   \includegraphics[width=200pt]{images/logo_ganeti.png}
  \end{center}
\end{frame}

\begin{frame}{Ganeti, qu'est-ce que c'est?}
\begin{itemize}
\item Un outil de gestion de cluster de serveur virtuel

\item Il utilise les hyperviseurs existants (XEN hypervisor,kvm)

\item Récupération rapide et simple, après des crashs physique

\item Utilisation de peu de ressources matériel

\item IaaS privés (L'infrastructure en tant que service)
\end{itemize}
\end{frame}

%\begin{frame}{Cluster de ganeti}
%\begin{center}
% \includegraphics[width=10cm,height=5cm]{images_presentation/ganeti_cluster.png}
%\end{center}
%\end{frame}


\begin{frame}{Background du projet}
\begin{block}{Historique}
  \begin{itemize}
  \item Projet financé par Google

  \item Open source depuis 2007 GPLv2

  \item Équipe Google basée en Suisse

  \item Liste de diffusion active et canal IRC

  \end{itemize}
\end{block}
\begin{block}{Organisations utilisant ganeti:}
  \begin{itemize}
  \item Google (utilisé dans leur infrastructure)

  \item Grnet.gr (Greek Research \& Technology Network)

  \item osuosl.org (Oregon State University Open Source Lab)
  \end{itemize}
\end{block}
\end{frame}

\begin{frame}{Composants}
\begin{center}
  \includegraphics[width=7cm,height=1.5cm]{images_presentation/module.png}
\end{center}
\begin{itemize}
\item Python et quelques modules
%\pause
\item Haskell
%\pause
\item DRBD
%\pause
\item LVM
%\pause
\item Hyperviseur
\end{itemize}
\end{frame}

\begin{frame}{Architechture}
\begin{center}
  \includegraphics[width=8cm,height=4cm]{images_presentation/archi1.png}
\end{center}
\end{frame}

\begin{frame}{Noeud}
\begin{itemize}
\item machine physique

\item La tolérance aux pannes n'est pas nécessaire

\item Ajouté / supprimé à volonté à partir du cluster

\item Aucune perte de données avec une perte de noeud
\end{itemize}
\end{frame}

\begin{frame}{Daemons}
\begin{itemize}
\item ganeti-noded : contrôler les ressources matérielles, qui fonctionne sur tous les noeuds

\item ganeti-confd : seulement fonctionnel sur le maître, et s'exécute sur tous les noeuds

\item ganeti-rapi : seulement sur l'API-HTTP  pour le cluster, fonctionne sur le maître

\item ganeti-masterd :  permet un contrôle du cluster, fonctionne sur le maître
\end{itemize}
\end{frame}

\begin{frame}{Instance}
\begin{center}
  \includegraphics[width=3cm,height=3cm]{images_presentation/instance.png}
\end{center}
\begin{itemize}
\item Machine virtuelle qui s'exécute sur le cluster

\item tolérant aux pannes / Haute disponibilité au sein du cluster
\end{itemize}
\end{frame}

\begin{frame}
Distributions suportées:\\
\begin{itemize}
\item Debian - trés bien supporté
%\pause
\item Gentoo - un support est apporté pour l'installation
%\pause
\item Ubuntu - devrait fonctionner
%\pause
\item CentOS - fonctionne mais quelques problèmes d'installation
\end{itemize}
\end{frame}

\begin{frame}{Planification réseau}
\begin{center}
  \includegraphics[width=3cm,height=3cm]{images_presentation/network.png}
\end{center}
\begin{block}{Ganeti supporte :}
\begin{itemize}
\item La connexion via un bridge
%\pause
\item Un réseau routé
%\pause
\item Noeuds sur un NAT privé
\end{itemize}
\end{block}
\end{frame}

\begin{frame}{Configuration du système d'exploitation}
\begin{itemize}
\item installation minimale du système
\pause
\item Volume du système de 20 Go minimum
\pause
\item Création d'un LVM pour les instances
\pause
\item 64bit est préférable
\pause
\item Matériel / logiciels similaires pour la configuration des nœuds
\end{itemize}
\end{frame}

\begin{frame}{Hyperviseur requis}
Obligatoire sur tous les nœuds
\begin{itemize}
\item Xen 3.0 et au-dessus \\ou
%\pause
\item KVM 0,11 et au-dessus
\end{itemize}
\end{frame}

\begin{frame}{Installation}
\begin{itemize}
\item Installation et configuration de ganeti
\pause
\item Mise en place de la haute disponibilité
\end{itemize}
\end{frame}

%\begin{frame}{Ce qui est installé}
%\begin{itemize}
%\item Bibliothèques Python sous le nom ganeti
%\pause
%\item Ensemble des programmes dans \emph{/usr/local/sbin ou /usr/sbin}
%\pause
%\item Ensemble d'outils dans \emph{lib/ganeti/} répertoire des outils
%\pause
%\item Scripts IAllocator sous \emph{lib/ganeti/outils\_annuaire}
%\pause
%\item Cron jobs nécessaires pour la maintenance du cluster
%\pause
%\item Script d'initialisation pour les démons ganeti
%\end{itemize}
%\end{frame}

\begin{frame}
  \begin{center}
   \huge{Démo}
  \end{center}
\end{frame}

\begin{frame}{Problèmes rencontrés}
 \begin{alertblock}{Problèmes}
   \begin{enumerate}
     \item Configuration automatique du réseau
%       \pause
     \item Très complet
       \pause
   \end{enumerate}
 \end{alertblock}
\pause
 \begin{exampleblock}{Solutions}
   \begin{enumerate}
     \item Utiliser les commandes de Ganeti
 %      \pause
     \item Plus de temps...
   \end{enumerate}
 \end{exampleblock}
\end{frame}



%\end{document}

\lstset{language=bash}
\chapter{OpenXENManager}
\section{Présentation}
XenseMaking Project développe un client lourd, ainsi qu'un client
web, pour manager XenServer. C'est un clone du XenCenter, qui fonctionne
avec Linux, BSD, Windows et MacOSX, alors que le XenCenter ne fonctionne
qu'avec Windows. OpenXenManager/OpenXenCenter un le client lourd qui
permet de manager XenServer.Il a été développé en Python avec pygtk
et gtk-vnc.
Les fonctionnalités actuellement implémentées sont les suivantes :
\begin{itemize}
\item monitoring des machines virtuelles - accès à la console des machines virtuelles
\item opérations d'administration (démarrage, arrèt, reboot, ...)
\item création de machines virtuelles
\end{itemize}
\section{Installation}
Pour l'installation nous avons besoin des paquets suivant:
\begin{lstlisting}
apt-get install subversion bzip2 python-glade2 python-gtk-vnc shared-mime-info graphviz
\end{lstlisting}
On télécharge la dernière version d'openxenmanager dans le dépot subversion
\begin{lstlisting}
svn co https://openxenmanager.svn.sourceforge.net/svnroot/openxenmanager openxenmanager
\end{lstlisting}
On se déplace dans le répertoire trunk:
\begin{lstlisting}
cd openxenmanager/trunk
\end{lstlisting}
Finalement on lance openxenmanager avec la commande suivante
\begin{lstlisting}
python window.py
\end{lstlisting}
Une interface graphique d'openxenmanger apparait.
\section{Utilisation}
Pour l'instant nous n'avons toujours pas réussi à utiliser ce logiciel. DU a diverses difficultées (pas de service distant sous squeeze et plus de xen lors de la migration sous unstable).
\chapter{Xen Cloud Platform}
\section{installation}
\begin{figure}
\begin{center}
\includegraphics[width=350pt]{images/1.png}
\end{center}
\caption{On choisit le  type de clavier}
\end{figure}
\begin{figure}
\begin{center}
\includegraphics[width=350pt]{images/2.png}
\end{center}
\caption{On accepte la licence}
\end{figure}
\begin{figure}
\begin{center}
\includegraphics[width=350pt]{images/3.png}
\end{center}
\caption{Choix du disque d'installation}
\end{figure}
\begin{figure}
\begin{center}
\includegraphics[width=350pt]{images/4.png}
\end{center}
\caption{Choix de la source d'installation}
\end{figure}
\begin{figure}
\begin{center}
\includegraphics[width=350pt]{images/5.png}
\end{center}
\caption{Paquets additionnels}
\end{figure}
\begin{figure}
\begin{center}
\includegraphics[width=350pt]{images/6.png}
\end{center}
\caption{Vérification de la source d'installation}
\end{figure}
\begin{figure}
\begin{center}
\includegraphics[width=350pt]{images/7.png}
\end{center}
\caption{Choix du password}
\end{figure}
\begin{figure}
\begin{center}
\includegraphics[width=350pt]{images/8.png}
\end{center}
\caption{Configuration du réseau}
\end{figure}
\begin{figure}
\begin{center}
\includegraphics[width=350pt]{images/9.png}
\end{center}
\caption{Configuration du DNS}
\end{figure}
\begin{figure}
\begin{center}
\includegraphics[width=350pt]{images/10.png}
\end{center}
\caption{Installation}
\end{figure}
\begin{figure}
\begin{center}
\includegraphics[width=350pt]{images/11.png}
\end{center}
\caption{Installation complète}
\end{figure}

\chapter{Installation de virt-manager}
\section{Pré-requis et considérations pour les hôtes}
Divers facteurs doivent être considérés avant de créer des hôtes virtualisés.
\subsubsection{Performance} 
Les hôtes virtualisés doivent être déployé et configuré en fonction de leurs tâches prévues. Certains systèmes (par exemple, les hôtes ou sont hébergés des serveur de base de données) ont besoin de performances plus élevées que d'habitude; Les hôtes peuvent exiger plus de CPU ou de mémoire attribué en fonction de leur rôle,et de l'utilisation futur qu'il pourrait avoir. projeté la charge du système.
\subsubsection{Stockage}
Certains hôtes peuvent avoir besoin d'une plus grande priorité d'accès au stockage, de disques plus rapides, ou peuvent exiger un accès exclusif à des zones de stockage. La quantité de stockage utilisée par les hôtes doit être régulièrement surveillée et prise en compte lors du déploiement et le maintien de stockage.
\subsubsection{Mise en réseau et l'infrastructure du réseau}
 En fonction de notre environnement, certains hôtes pourraient exiger des liens réseau plus rapides que d'autres hôtes. La bande passante ou de latence sont souvent des facteurs à prendre en compte lors du déploiement et le maintenance des hôtes.
\section{Installation côté serveur}
Cette partie est facile, un simple apt-get install suffit. Nous installons le paquet qui communique avec Xen et remonte les informations au client virt-manager.
\begin{lstlisting}
apt-get install libvirt-bin
\end{lstlisting} 
Du coté de Xen, nous devons vérifier qu’il peut communiquer avec libvirt.

Libvirt accède aux données de Xen via un socket unix. La configuration consiste à activer cette option dans Xen et à relancer les services.
Nous éviterons ainsi l’erreur libvirtError: internal error failed to connect to xend dont on trouve peu d’explication sur le net.

On édite le fichier de configuration xen
\begin{lstlisting} 
nano /etc/xen/xend-config.sxp
\end{lstlisting}
 on active la ligne suivante
\begin{lstlisting} 
(xend-unix-server yes)
\end{lstlisting}
 Enfin on relance le service xen avec /etc/init.d/xend restart

\section{Installation côté client}
Pour gérer nos serveurs, nous installons virt-manager avec la commande suivante:
\begin{lstlisting} 
apt-get install virt-manager
\end{lstlisting}
\Chapter{Création d'hôtes virtualisés avec virt-manager}


1)Pour commencer on démmarre virt-manager, puis on lance le gestionnaire de machines virtuelles à partir du menu en cliquant sur l'icone en forme de pc.

\begin{figure}
\begin{center}
\includegraphics[width=350pt]{images/virt.jpg}
\end{center}
\caption{Interface de virt-manager}
\end{figure}



3) La fenêtre du gestionnaire de machine virtuelle nous autorise à en créer de nouvelles.
On clique sur création de nouvelle machine virtuelle pour faire apparaître l'assistant qui va nous aider pour élaborer notre hôte.
L'assistant décompose la création en cinq étapes:
-La localisation et la configuration des supports d'installation
-La configuration de la mémoire et les options de CPU
-La configuration du stockage de l'invité
-La configuration réseau, l'architecture, et d'autres paramètres matériels

Le processus de création d'hôte commence avec la selection d'un nom et le type d'installation.
\begin{figure}
\begin{center}
\includegraphics[width=350pt]{images/nommachine.jpg}
\end{center}
\caption{Choix du nom de la nouvelle machine virtuelle}
\end{figure}
-Local install media(ISO image or CDROM)
Cette méthode utilise un CD-ROM,DVD ou une image iso.

-Network install
Cette méthode utilise le réseau pour installer le système d'exploitation.

-Import existing disk image
Cette méthode nous permet de créer un nouvelle hôte et d'y importer une image disque.

La prochaine étape consiste à configurer l'installation.
On configure le type de système d'exploitation et sa version qui sera installé, cela dépend de la méthode d'installation que l'on a choisie.
\begin{figure}
\begin{center}
\includegraphics[width=300pt]{images/iso.jpg}
\end{center}
\caption{Choix du nom de la nouvelle machine virtuelle}
\end{figure}



Configuration du CPU et de la mémoire
La prochaine étape consiste à configurer le nombre de CPU et la quantité de mémoire à allouer à la machine virtuelle. L'assistant indique le nombre de processeurs et la quantité de mémoire que l'on peut lui allouer.
\begin{figure}
\begin{center}
\includegraphics[width=300pt]{images/cpu.jpg}
\end{center}
\caption{Choix du nom de la nouvelle machine virtuelle}
\end{figure}

Configuration de l'espace de stockage
\begin{figure}
\begin{center}
\includegraphics[width=300pt]{images/Storage.jpg}
\end{center}
\caption{Choix du nom de la nouvelle machine virtuelle}
\end{figure}


 Si l'on a choisi d'importer une image de disque existante au cours de la première étape, virt-manager va sauter cette étape.
On doit attribuer un espace suffisant pour notre machine virtuelle et toutes les applications que l'hôte a besoin.

Configuration finale
On vérifie les paramètres de la machine virtuelle et on clique sur Terminer lorsqu'on est satisfait, cela permettra de créer l'hôte avec les paramètres réseau par défaut, le type de virtualisation, et l'architecture.
\begin{figure}
\begin{center}
\includegraphics[width=300pt]{images/reseau.jpg}
\end{center}
\caption{Choix du nom de la nouvelle machine virtuelle}
\end{figure}


\appendix
\chapter{Sources}
\begin{description}
\item[www.grid5000.fr : ]le wiki disponible sur le site internet de grid5000 fut principale source de renseignements pour le démarrage du projet.
\item[www.loria.fr/lnussbau/ : ]nous avons pu y consulter des anciens projets sur Grid5000 ce qui nous a permis d'avoir un premier aperçu de ses possibilités.
\end{description}
\chapter{Scripts}
\section{réservation}
\lstset{language=ruby}
\begin{lstlisting}
puts "-----------------------------------------"
puts "Souhaitez vous reserver des noeuds?(y/n)"
loop do
  test = gets.chomp
  if test.eql?("n")
    puts "#####################"
    puts "#sortie du programme#"
    puts "#####################"
    break;
  end
  if test.eql?("y")
    #Reservation de machines                                                 
    puts "---------Script de reservation-----------"
    puts "Choisir un nombre de noeud:"
    noeuds = gets.chomp.to_i
    puts "Choisir un temps de reservation(HH:MM:SS):"
    temps = gets
    puts "\nVous avez reserve #{noeuds} noeuds 
	  pour une duree de #{temps}"
    puts "-----------------------------------------"
    exec "oarsub -I -t deploy -n'virtu' -l 
	  slash_22=1+nodes=#{noeuds},walltime=#{temps}"
    break;
  end
end
\end{lstlisting}

\newpage
\input{parties_rapport/script_deploy.tex}
\input{parties_rapport/script_config.tex}
\newpage
\section{Listing d'unne installation de Ganetti}
\begin{lstlisting}
#!/bin/bash
#passage en wheezy
rm /etc/apt/sources.list
echo "## wheezy security" > /etc/apt/sources.list
echo "deb http://security.debian.org/ wheezy/updates main contrib non-free" >> /etc/apt/sources.list
echo "deb-src http://security.debian.org/ wheezy/updates main contrib non-free" >> /etc/apt/sources.list
echo " " >> /etc/apt/sources.list
echo "#wheezy" >> /etc/apt/sources.list
echo "deb http://ftp.fr.debian.org/debian/ wheezy main contrib non-free" >> /etc/apt/sources.list
echo "deb-src http://ftp.fr.debian.org/debian/ wheezy main contrib non-free" >> /etc/apt/sources.list
#Installation de ganeti 
apt-get update
apt-get dist-upgrade -y --force-yes
apt-get install -y --force-yes ganeti2 ganeti-htools ganeti-instance-debootstrap
echo "Ajout du node dans /etc/hosts"
hostname=`cat /etc/hostname`
#recuperation des variables
#ip du node
ifconfig eth0 > troll
ipnode=`head -2 troll | tail -1 | cut -d':' -f2 | cut -d' ' -f1`
echo \$ipnode \$hostname >> /etc/hosts
#ip du broadcast
ipbroadcast=`head -2 troll | tail -1 | cut -d'B' -f2 | cut -d':' -f2 | cut -d' ' -f1`
#ip du masque de sous reseau
ipmask=`head -2 troll | tail -1 | cut -d'M' -f2 | cut -d':' -f2 | cut -d' ' -f1`
#ip du reseau
ipnetwork=`head -1 ipnetwork`
#ip de la passerelle
ipgateway=`head -1 ipgateway`
#ajout de cluster1 dans dans /etc/hosts
echo "ajout de cluster1 dans /etc/hosts"
ipcluster=`cat ipcluster`
echo \$ipcluster "cluster1" >> /etc/hosts
#Dans /boot/ creer des liens symboliques :
ln -s /boot/vmlinuz-2.6.32-5-xen-amd64 /boot/vmlinuz-2.6.xenU
ln -s /boot/initrd.img-2.6.32-5-xen-amd64 /boot/initrd.img-2.6.xenU
#Pour le moment changera surement.
echo "creation du LVM"
umount /dev/sda5
pvcreate /dev/sda5
vgcreate xenvg /dev/sda5
#Creation du bridge xen-br0
echo " " >> /etc/network/interfaces
echo "auto xen-br0"  >> /etc/network/interfaces
echo "iface xen-br0 inet static" >> /etc/network/interfaces 
echo "address" \$ipnode >> /etc/network/interfaces
echo "netmask " \$ipmask >> /etc/network/interfaces
echo "network" \$ipnetwork >> /etc/network/interfaces
echo "gateway"  \$ipgateway >> /etc/network/interfaces
echo "broadcast" \$ipbroadcast>> /etc/network/interfaces
echo "bridge_ports eth0" >> /etc/network/interfaces
echo "bridge_stp off" >> /etc/network/interfaces
echo "bridge_fd 0" >> /etc/network/interfaces
#Suppression des ligne de eth0
sed -i '9d' /etc/network/interfaces
sed -i '9d' /etc/network/interfaces
#supression des fichier temporaires
rm troll ipcluster  ipgateway  ipnetwork
#initialisation du cluster
gnt-cluster init --no-drbd-storage cluster1
#ajouter le node 
gnt-node add \$hostname
#et verifier
gnt-node list
\end{lstlisting}


\chapter{Glossaire}
\begin{description}
\item[]
\item[]
\end{description}
\end{document}
