\documentclass{beamer}
\usepackage[utf8]{inputenc}
\usepackage{colortbl}
\usepackage{multimedia}
\usetheme{JuanLesPins}
\mode<presentation>
\useoutertheme{smoothtree}
\usecolortheme{whale}
\usecolortheme{orchid}
\useinnertheme[shadow=true]{rounded}
% profondeur table des matieres
\setcounter{tocdepth}{1}
\setbeamerfont{block title}{size={}}
\title{Projet Tuteuré \\Gestion centralisée de machines virtuelles}
\author{Augustin Bocca Julien Tournois\\Sébastien Michaux Mathieu Lamouroux}
\institute{IUT de Nancy Charlemagne}
%\addtobeamertemplate{footline}{\insertframenumber/\inserttotalframenumber}
\begin{document}

\AtBeginSection[]
{
  \begin{frame}<beamer>
    \frametitle{Plan}
    \tableofcontents[currentsection,currentsubsection,hideothersubsections]
  \end{frame}
}

\begin{frame}
  \titlepage
\end{frame}

\begin{frame}
    \frametitle{Plan}
    \tableofcontents
\end{frame}

%%%%%%%%%%%%%%%%%%%%%%%%%%%%%%%%%%%%%%%%%%%%%%%%%%%%%%%%%%%%
\section{Le contexte}
\subsection{Le projet}
\begin{frame}{Le projet tuteuré}
 %%%% trouver une boite
\begin{block}{Intitulé}
Mettre en place, évaluer et comparer différents outils de gestion centralisée de machines virtuelles.
\end{block}
\pause
\begin{block}{Résultats attendus}
  \begin{itemize}
    \item guide d'installation et d'utilisation synthétique
\pause
    \item scripts
\pause
    \item démos à grande échelle sur grid5000
\pause
    \item avis critique
   \end{itemize}
\end{block}
\end{frame}

\subsection{Grid5000}
\begin{frame}{La plateforme Grid5000}
%%%%%%%%% PHOTO
\begin{block}{Vue d'ensemble}
  \begin{itemize}
    \item Grille Informatique
\pause
    \item Dix sites en France
\pause
    \item Réliés par RENATER
\pause
    \item Objectif scientifique
  \end{itemize}
\end{block}
\end{frame}

\begin{frame}{Architecture type d'un site}
  \begin{center}
    \includegraphics[width=180pt]{images_presentation/archi.png}
  \end{center}
\end{frame}

\begin{frame}{Connexion à un site}
  \includegraphics[width=330pt]{images_presentation/plan_site.png}
\end{frame}


%%%%%%%%%%%%%%%%%%%%%%%%%%%%%%%%%%%%%%%%%%%%%%%%%%%%%%%%%%%%
\section{La virtualisation}
\subsection{Historique}
\begin{frame}{Il était une fois la virtualisation...}
  \begin{description}
    \item[1960 : ] inventée par IBM pour optimiser l'utilisation du matériel sur les serveurs
\pause
    \item[1990 : ] VMware porte le concept sur les plateformes x86
\pause
    \item[Aujourd'hui : ] VMware se positionne en tant que leader du marché.
  \end{description}
\end{frame}

\subsection{Enjeux}
\begin{frame}{Pourquoi virtualisé?}
  Et c'est facile.
\end{frame}

%%%%%%%%%%%%%%%%%%%%%%%%%%%%%%%%%%%%%%%%%%%%%%%%%%%%%%%%%%%%
\section{Logiciels testés}
\subsection{Ganeti}
%%%%%%%% LOGO
%\documentclass{beamer}

%\usepackage[utf8x]{inputenc}
%\usepackage{default}
%\usepackage{graphicx}

%\begin{document}

\begin{frame}
  \begin{center}
   \includegraphics[width=200pt]{images/logo_ganeti.png}
  \end{center}
\end{frame}

\begin{frame}{Ganeti, qu'est-ce que c'est?}
\begin{itemize}
\item Un outil de gestion de cluster de serveur virtuel

\item Il utilise les hyperviseurs existants (XEN hypervisor,kvm)

\item Récupération rapide et simple, après des crashs physique

\item Utilisation de peu de ressources matériel

\item IaaS privés (L'infrastructure en tant que service)
\end{itemize}
\end{frame}

%\begin{frame}{Cluster de ganeti}
%\begin{center}
% \includegraphics[width=10cm,height=5cm]{images_presentation/ganeti_cluster.png}
%\end{center}
%\end{frame}


\begin{frame}{Background du projet}
\begin{block}{Historique}
  \begin{itemize}
  \item Projet financé par Google

  \item Open source depuis 2007 GPLv2

  \item Équipe Google basée en Suisse

  \item Liste de diffusion active et canal IRC

  \end{itemize}
\end{block}
\begin{block}{Organisations utilisant ganeti:}
  \begin{itemize}
  \item Google (utilisé dans leur infrastructure)

  \item Grnet.gr (Greek Research \& Technology Network)

  \item osuosl.org (Oregon State University Open Source Lab)
  \end{itemize}
\end{block}
\end{frame}

\begin{frame}{Composants}
\begin{center}
  \includegraphics[width=7cm,height=1.5cm]{images_presentation/module.png}
\end{center}
\begin{itemize}
\item Python et quelques modules
%\pause
\item Haskell
%\pause
\item DRBD
%\pause
\item LVM
%\pause
\item Hyperviseur
\end{itemize}
\end{frame}

\begin{frame}{Architechture}
\begin{center}
  \includegraphics[width=8cm,height=4cm]{images_presentation/archi1.png}
\end{center}
\end{frame}

\begin{frame}{Noeud}
\begin{itemize}
\item machine physique

\item La tolérance aux pannes n'est pas nécessaire

\item Ajouté / supprimé à volonté à partir du cluster

\item Aucune perte de données avec une perte de noeud
\end{itemize}
\end{frame}

\begin{frame}{Daemons}
\begin{itemize}
\item ganeti-noded : contrôler les ressources matérielles, qui fonctionne sur tous les noeuds

\item ganeti-confd : seulement fonctionnel sur le maître, et s'exécute sur tous les noeuds

\item ganeti-rapi : seulement sur l'API-HTTP  pour le cluster, fonctionne sur le maître

\item ganeti-masterd :  permet un contrôle du cluster, fonctionne sur le maître
\end{itemize}
\end{frame}

\begin{frame}{Instance}
\begin{center}
  \includegraphics[width=3cm,height=3cm]{images_presentation/instance.png}
\end{center}
\begin{itemize}
\item Machine virtuelle qui s'exécute sur le cluster

\item tolérant aux pannes / Haute disponibilité au sein du cluster
\end{itemize}
\end{frame}

\begin{frame}
Distributions suportées:\\
\begin{itemize}
\item Debian - trés bien supporté
%\pause
\item Gentoo - un support est apporté pour l'installation
%\pause
\item Ubuntu - devrait fonctionner
%\pause
\item CentOS - fonctionne mais quelques problèmes d'installation
\end{itemize}
\end{frame}

\begin{frame}{Planification réseau}
\begin{center}
  \includegraphics[width=3cm,height=3cm]{images_presentation/network.png}
\end{center}
\begin{block}{Ganeti supporte :}
\begin{itemize}
\item La connexion via un bridge
%\pause
\item Un réseau routé
%\pause
\item Noeuds sur un NAT privé
\end{itemize}
\end{block}
\end{frame}

\begin{frame}{Configuration du système d'exploitation}
\begin{itemize}
\item installation minimale du système
\pause
\item Volume du système de 20 Go minimum
\pause
\item Création d'un LVM pour les instances
\pause
\item 64bit est préférable
\pause
\item Matériel / logiciels similaires pour la configuration des nœuds
\end{itemize}
\end{frame}

\begin{frame}{Hyperviseur requis}
Obligatoire sur tous les nœuds
\begin{itemize}
\item Xen 3.0 et au-dessus \\ou
%\pause
\item KVM 0,11 et au-dessus
\end{itemize}
\end{frame}

\begin{frame}{Installation}
\begin{itemize}
\item Installation et configuration de ganeti
\pause
\item Mise en place de la haute disponibilité
\end{itemize}
\end{frame}

%\begin{frame}{Ce qui est installé}
%\begin{itemize}
%\item Bibliothèques Python sous le nom ganeti
%\pause
%\item Ensemble des programmes dans \emph{/usr/local/sbin ou /usr/sbin}
%\pause
%\item Ensemble d'outils dans \emph{lib/ganeti/} répertoire des outils
%\pause
%\item Scripts IAllocator sous \emph{lib/ganeti/outils\_annuaire}
%\pause
%\item Cron jobs nécessaires pour la maintenance du cluster
%\pause
%\item Script d'initialisation pour les démons ganeti
%\end{itemize}
%\end{frame}

\begin{frame}
  \begin{center}
   \huge{Démo}
  \end{center}
\end{frame}

\begin{frame}{Problèmes rencontrés}
 \begin{alertblock}{Problèmes}
   \begin{enumerate}
     \item Configuration automatique du réseau
%       \pause
     \item Très complet
       \pause
   \end{enumerate}
 \end{alertblock}
\pause
 \begin{exampleblock}{Solutions}
   \begin{enumerate}
     \item Utiliser les commandes de Ganeti
 %      \pause
     \item Plus de temps...
   \end{enumerate}
 \end{exampleblock}
\end{frame}



%\end{document}

\subsection{Virt-Manager}
%%%%%%% LOGO
\documentclass{article}
\usepackage[utf8]{inputenc}
\usepackage[frenchb]{babel}
\usepackage{listings}
\usepackage{color} 
\definecolor{hellgelb}{rgb}{1,1,0.8}
\definecolor{colKeys}{rgb}{0,0,1}
\definecolor{colIdentifier}{rgb}{0,0,0}
\definecolor{colComments}{rgb}{1,0,0}
\definecolor{colString}{rgb}{0,0.5,0}
\lstset{%
language={},%
morekeywords={} %
}
\lstset{%
float=hbp,%
basicstyle=\ttfamily\small, %
identifierstyle=\color{colIdentifier}, %
keywordstyle=\color{colKeys}, %
stringstyle=\color{colString}, %
commentstyle=\color{colComments}, %
columns=flexible, %
tabsize=2, %
frame=single, %
extendedchars=true, %
showspaces=false, %
showstringspaces=false, %
numbers=left, %
numberstyle=\tiny, %
breaklines=true, %
backgroundcolor=\color{hellgelb}, %
breakautoindent=true, %
captionpos=b%
}
\begin{document}
\title{Virt-Manager}
\author{Mathieu Lamouroux - Sébastien Michaux}
\date{5 Mars 2012}
\maketitle

\begin{abstract}
Rapport sur le test de déploiement de virt-manager sur Grid5000.\\
Pour cela nous avons réservé 10 noeuds sur Grid5000 : un sera dévoué à virt-manager, les autres contiendront des machines virtuelles
\end{abstract}
\section{Contexte}
Machines déployées : Debian Squeeze
griffon-50.nancy.grid5000.fr -> MASTER

griffon-51.nancy.grid5000.fr
griffon-53.nancy.grid5000.fr
griffon-65.nancy.grid5000.fr
griffon-68.nancy.grid5000.fr
griffon-70.nancy.grid5000.fr
griffon-77.nancy.grid5000.fr
griffon-79.nancy.grid5000.fr
griffon-87.nancy.grid5000.fr
griffon-90.nancy.grid5000.fr

Nous allons déployer 7 machines virtuelles sur chacun des noeuds.
\section{Déploiement}
Paquets mis à jour sur les noeuds
\begin{lstlisting}
root@griffon-50:~# apt-get upgrade
Reading package lists... Done
Building dependency tree       
Reading state information... Done
The following packages will be upgraded:
  base-files bzip2 curl dpkg dpkg-dev file firmware-linux-free ia32-libs
  lib32bz2-1.0 libbz2-1.0 libc-bin libc-dev-bin libc6 libc6-dev libc6-i386
  libcurl3 libdpkg-perl libgnutls26 libgssapi-krb5-2 libk5crypto3 libkrb5-3
  libkrb5support0 libmagic1 libpng12-0 librpm1 librpmbuild1 librpmio1
  libssl0.9.8 libxenstore3.0 libxml2 linux-base linux-headers-2.6.32-5-amd64
  linux-headers-2.6.32-5-common linux-image-2.6.32-5-xen-amd64 linux-libc-dev
  locales module-init-tools openssl perl perl-base perl-modules rpm rpm-common
  rpm2cpio tzdata xen-hypervisor-4.0-amd64 xen-utils-4.0 xenstore-utils
48 upgraded, 0 newly installed, 0 to remove and 0 not upgraded.
Need to get 110 MB of archives.
After this operation, 434 kB disk space will be freed.
Do you want to continue [Y/n]? 
\end{lstlisting}
Installation de libvirt
\begin{lstlisting}
root@griffon-50:~# apt-get install libvirt-bin
Reading package lists... Done
Building dependency tree       
Reading state information... Done
The following extra packages will be installed:
  dnsmasq-base ebtables libaio1 libasyncns0 libbluetooth3 libbrlapi0.5
  libcap-ng0 libcurl3-gnutls libdirectfb-1.2-9 libflac8 libnl1 libogg0
  libparted0debian1 libpciaccess0 libpulse0 libsdl1.2debian
  libsdl1.2debian-alsa libsndfile1 libsvga1 libts-0.0-0 libvdeplug2 libvirt0
  libvorbis0a libvorbisenc2 libx86-1 libxml2-utils libxtst6 netcat-openbsd
  qemu-kvm tsconf
Suggested packages:
  parted nparted libparted0-dev libparted0-i18n pulseaudio policykit-1 vde2
  samba
The following NEW packages will be installed:
  dnsmasq-base ebtables libaio1 libasyncns0 libbluetooth3 libbrlapi0.5
  libcap-ng0 libcurl3-gnutls libdirectfb-1.2-9 libflac8 libnl1 libogg0
  libparted0debian1 libpciaccess0 libpulse0 libsdl1.2debian
  libsdl1.2debian-alsa libsndfile1 libsvga1 libts-0.0-0 libvdeplug2
  libvirt-bin libvirt0 libvorbis0a libvorbisenc2 libx86-1 libxml2-utils
  libxtst6 netcat-openbsd qemu-kvm tsconf
0 upgraded, 31 newly installed, 0 to remove and 0 not upgraded.
Need to get 8,044 kB of archives.
After this operation, 21.7 MB of additional disk space will be used.
Do you want to continue [Y/n]? 
\end{lstlisting}
Il faut ensuite activer l'utilisation des sockets Unix pour permettre à libvirt de communiquer avec xen. Ceci se fait dans le fichier \emph{/etc/xen/xend-config.sxp} en ajoutant ou activant la ligne \emph{(xend-unix-server yes)} puis en redémarant le service xen.\\
On s'assure que le service est bien actif
\begin{lstlisting}
root@griffon-50:~# virsh version
Compiled against library: libvir 0.8.3
Using library: libvir 0.8.3
Using API: Xen 3.0.1
Running hypervisor: Xen 4.0.0
\end{lstlisting}
Ensuite on installe le client virsh (uniquement sur griffon-50 dans notre cas)
\begin{lstlisting}
root@griffon-50:~# apt-get install virt-manager
Reading package lists... Done
Building dependency tree       
Reading state information... Done
The following extra packages will be installed:
  consolekit dbus-x11 dosfstools esound-common etherboot-qemu gconf2
  gconf2-common gnome-icon-theme gnome-mime-data gvfs hdparm libart-2.0-2
  libatasmart4 libaudiofile0 libavahi-glib1 libblas3gf libbonobo2-0
  libbonobo2-common libbonoboui2-0 libbonoboui2-common libck-connector0
  libcroco3 libdbus-glib-1-2 libeggdbus-1-0 libesd0 libfam0 libffi5 libgail18
  libgconf2-4 libgdu0 libgfortran3 libglade2-0 libgnome-keyring0 libgnome2-0
  libgnome2-common libgnomecanvas2-0 libgnomecanvas2-common libgnomeui-0
  libgnomeui-common libgnomevfs2-0 libgnomevfs2-common libgnomevfs2-extra
  libgsf-1-114 libgsf-1-common libgtk-vnc-1.0-0 libgudev-1.0-0 libhal-storage1
  libhal1 libidl0 liblapack3gf libntfs-3g75 libntfs10 liborbit2
  libpam-ck-connector libpcap0.8 libpolkit-agent-1-0 libpolkit-backend-1-0
  libpolkit-gobject-1-0 librsvg2-2 librsvg2-common libsgutils2-2 libsmbclient
  libtalloc2 libvde0 libvte-common libvte9 libwbclient0 libxaw7 libxcb-atom1
  libxmu6 libxpm4 libxt6 libxv1 libxxf86dga1 mtools ntfs-3g ntfsprogs
  openbios-ppc openbios-sparc openhackware policykit-1 policykit-1-gnome
  python-cairo python-dbus python-gconf python-glade2 python-gnome2
  python-gobject python-gtk-vnc python-gtk2 python-libvirt python-libxml2
  python-numpy python-pycurl python-pyorbit python-urlgrabber python-vte qemu
  qemu-keymaps qemu-system qemu-user qemu-utils seabios udisks vde2 vgabios
  virt-viewer virtinst x11-utils
Suggested packages:
  esound-clients gconf-defaults-service gvfs-backends apmd libbonobo2-bin
  esound fam gnome-keyring desktop-base libgnomevfs2-bin librsvg2-bin
  sg3-utils floppyd python-dbus-doc python-dbus-dbg python-gnome2-doc
  python-gtk2-doc python-gobject-dbg python-numpy-doc python-numpy-dbg
  python-nose python-dev gfortran libcurl4-gnutls-dev python-pycurl-dbg
  qemu-user-static samba xfsprogs reiserfsprogs mdadm cryptsetup vde2-cryptcab
  ssh-askpass mesa-utils
The following NEW packages will be installed:
  consolekit dbus-x11 dosfstools esound-common etherboot-qemu gconf2
  gconf2-common gnome-icon-theme gnome-mime-data gvfs hdparm libart-2.0-2
  libatasmart4 libaudiofile0 libavahi-glib1 libblas3gf libbonobo2-0
  libbonobo2-common libbonoboui2-0 libbonoboui2-common libck-connector0
  libcroco3 libdbus-glib-1-2 libeggdbus-1-0 libesd0 libfam0 libffi5 libgail18
  libgconf2-4 libgdu0 libgfortran3 libglade2-0 libgnome-keyring0 libgnome2-0
  libgnome2-common libgnomecanvas2-0 libgnomecanvas2-common libgnomeui-0
  libgnomeui-common libgnomevfs2-0 libgnomevfs2-common libgnomevfs2-extra
  libgsf-1-114 libgsf-1-common libgtk-vnc-1.0-0 libgudev-1.0-0 libhal-storage1
  libhal1 libidl0 liblapack3gf libntfs-3g75 libntfs10 liborbit2
  libpam-ck-connector libpcap0.8 libpolkit-agent-1-0 libpolkit-backend-1-0
  libpolkit-gobject-1-0 librsvg2-2 librsvg2-common libsgutils2-2 libsmbclient
  libtalloc2 libvde0 libvte-common libvte9 libwbclient0 libxaw7 libxcb-atom1
  libxmu6 libxpm4 libxt6 libxv1 libxxf86dga1 mtools ntfs-3g ntfsprogs
  openbios-ppc openbios-sparc openhackware policykit-1 policykit-1-gnome
  python-cairo python-dbus python-gconf python-glade2 python-gnome2
  python-gobject python-gtk-vnc python-gtk2 python-libvirt python-libxml2
  python-numpy python-pycurl python-pyorbit python-urlgrabber python-vte qemu
  qemu-keymaps qemu-system qemu-user qemu-utils seabios udisks vde2 vgabios
  virt-manager virt-viewer virtinst x11-utils
0 upgraded, 110 newly installed, 0 to remove and 0 not upgraded.
Need to get 63.5 MB of archives.
After this operation, 186 MB of additional disk space will be used.
Do you want to continue [Y/n]? 
root@griffon-50:~# apt-get install virt-manager
Reading package lists... Done
Building dependency tree       
Reading state information... Done
The following extra packages will be installed:
  consolekit dbus-x11 dosfstools esound-common etherboot-qemu gconf2
  gconf2-common gnome-icon-theme gnome-mime-data gvfs hdparm libart-2.0-2
  libatasmart4 libaudiofile0 libavahi-glib1 libblas3gf libbonobo2-0
  libbonobo2-common libbonoboui2-0 libbonoboui2-common libck-connector0
  libcroco3 libdbus-glib-1-2 libeggdbus-1-0 libesd0 libfam0 libffi5 libgail18
  libgconf2-4 libgdu0 libgfortran3 libglade2-0 libgnome-keyring0 libgnome2-0
  libgnome2-common libgnomecanvas2-0 libgnomecanvas2-common libgnomeui-0
  libgnomeui-common libgnomevfs2-0 libgnomevfs2-common libgnomevfs2-extra
  libgsf-1-114 libgsf-1-common libgtk-vnc-1.0-0 libgudev-1.0-0 libhal-storage1
  libhal1 libidl0 liblapack3gf libntfs-3g75 libntfs10 liborbit2
  libpam-ck-connector libpcap0.8 libpolkit-agent-1-0 libpolkit-backend-1-0
  libpolkit-gobject-1-0 librsvg2-2 librsvg2-common libsgutils2-2 libsmbclient
  libtalloc2 libvde0 libvte-common libvte9 libwbclient0 libxaw7 libxcb-atom1
  libxmu6 libxpm4 libxt6 libxv1 libxxf86dga1 mtools ntfs-3g ntfsprogs
  openbios-ppc openbios-sparc openhackware policykit-1 policykit-1-gnome
  python-cairo python-dbus python-gconf python-glade2 python-gnome2
  python-gobject python-gtk-vnc python-gtk2 python-libvirt python-libxml2
  python-numpy python-pycurl python-pyorbit python-urlgrabber python-vte qemu
  qemu-keymaps qemu-system qemu-user qemu-utils seabios udisks vde2 vgabios
  virt-viewer virtinst x11-utils
Suggested packages:
  esound-clients gconf-defaults-service gvfs-backends apmd libbonobo2-bin
  esound fam gnome-keyring desktop-base libgnomevfs2-bin librsvg2-bin
  sg3-utils floppyd python-dbus-doc python-dbus-dbg python-gnome2-doc
  python-gtk2-doc python-gobject-dbg python-numpy-doc python-numpy-dbg
  python-nose python-dev gfortran libcurl4-gnutls-dev python-pycurl-dbg
  qemu-user-static samba xfsprogs reiserfsprogs mdadm cryptsetup vde2-cryptcab
  ssh-askpass mesa-utils
The following NEW packages will be installed:
  consolekit dbus-x11 dosfstools esound-common etherboot-qemu gconf2
  gconf2-common gnome-icon-theme gnome-mime-data gvfs hdparm libart-2.0-2
  libatasmart4 libaudiofile0 libavahi-glib1 libblas3gf libbonobo2-0
  libbonobo2-common libbonoboui2-0 libbonoboui2-common libck-connector0
  libcroco3 libdbus-glib-1-2 libeggdbus-1-0 libesd0 libfam0 libffi5 libgail18
  libgconf2-4 libgdu0 libgfortran3 libglade2-0 libgnome-keyring0 libgnome2-0
  libgnome2-common libgnomecanvas2-0 libgnomecanvas2-common libgnomeui-0
  libgnomeui-common libgnomevfs2-0 libgnomevfs2-common libgnomevfs2-extra
  libgsf-1-114 libgsf-1-common libgtk-vnc-1.0-0 libgudev-1.0-0 libhal-storage1
  libhal1 libidl0 liblapack3gf libntfs-3g75 libntfs10 liborbit2
  libpam-ck-connector libpcap0.8 libpolkit-agent-1-0 libpolkit-backend-1-0
  libpolkit-gobject-1-0 librsvg2-2 librsvg2-common libsgutils2-2 libsmbclient
  libtalloc2 libvde0 libvte-common libvte9 libwbclient0 libxaw7 libxcb-atom1
  libxmu6 libxpm4 libxt6 libxv1 libxxf86dga1 mtools ntfs-3g ntfsprogs
  openbios-ppc openbios-sparc openhackware policykit-1 policykit-1-gnome
  python-cairo python-dbus python-gconf python-glade2 python-gnome2
  python-gobject python-gtk-vnc python-gtk2 python-libvirt python-libxml2
  python-numpy python-pycurl python-pyorbit python-urlgrabber python-vte qemu
  qemu-keymaps qemu-system qemu-user qemu-utils seabios udisks vde2 vgabios
  virt-manager virt-viewer virtinst x11-utils
0 upgraded, 110 newly installed, 0 to remove and 0 not upgraded.
Need to get 63.5 MB of archives.
After this operation, 186 MB of additional disk space will be used.
Do you want to continue [Y/n]? 
\end{lstlisting}

\section{Installation}

\section{}

\end{document}

%%%%%%%%%%%%%%%%%%%%%%%%%%%%%%%%%%%%%%%%%%%%%%%%%%%%%%%%%%%%
\section{Logiciels non-testés}

\subsection{Archipel}
%%%%% LOGO
\begin{frame}{Présentation}
  Et c'est facile.
\end{frame}
\begin{frame}{Installation}
  Et c'est facile.
\end{frame}
\begin{frame}{Utilisation}
  Et c'est facile.
\end{frame}

\subsection{OpenXenManager}
\begin{frame}
  \begin{center}
   \huge{OpenXenManager}
  \end{center}
\end{frame}

\begin{frame}{Présentation}
  \begin{itemize}
    \item développé par XenseMaking Project 
      \pause
    \item gérer des environnements XenServer
      \pause
    \item clone de XenCenter
  \end{itemize}
\end{frame}

\begin{frame}{Installation}
  \begin{itemize}
    \item Disponible dans les dépôts
      \pause
    \item Téléchargement depuis le site du projet
      \pause
    \item Répertoire subversion
  \end{itemize}
\end{frame}

\begin{frame}{Problèmes rencontrés}
 \begin{alertblock}{Problèmes}
   \begin{itemize}
     \item Communications avec xend
       \pause
     \item Installation de XCP
       \pause
     \item Mise à jour du système
   \end{itemize}
 \end{alertblock}
\pause
 \begin{exampleblock}{Solutions envisagées}
   \begin{itemize}
     \item Configuration de xend
       \pause
     \item Installation avec virt-manager
   \end{itemize}
 \end{exampleblock}
\end{frame}


%%%%% LOGO


%%%%%%%%%%%%%%%%%%%%%%%%%%%%%%%%%%%%%%%%%%%%%%%%%%%%%%%%%%%%
\section{Conclusion}
\subsection{Comparatif}
\begin{frame}{Comparaison des solutions testées}
\begin{center}
\begin{tabular}{|c|c|c|c|c|}
\hline
 & OXM & Ganeti & Virt-Manager & Archipel \tabularnewline
\hline
Documentation & \includegraphics[width=10pt]{images_presentation/bad.png} & \includegraphics[width=10pt]{images_presentation/ok.png} & \includegraphics[width=10pt]{images_presentation/moyen.png} & \includegraphics[width=10pt]{images_presentation/ok.png} \tabularnewline
\hline
Communauté &\includegraphics[width=10pt]{images_presentation/moyen.png} & \includegraphics[width=10pt]{images_presentation/ok.png}& \includegraphics[width=10pt]{images_presentation/moyen.png}& \includegraphics[width=10pt]{images_presentation/moyen.png}\tabularnewline
\hline
Maturité &\includegraphics[width=10pt]{images_presentation/moyen.png}  & \includegraphics[width=10pt]{images_presentation/ok.png} & \includegraphics[width=10pt]{images_presentation/ok.png}& \includegraphics[width=10pt]{images_presentation/moyen.png} \tabularnewline
\hline
Installation & \includegraphics[width=10pt]{images_presentation/moyen.png}&\includegraphics[width=10pt]{images_presentation/moyen.png} & \includegraphics[width=10pt]{images_presentation/ok.png}& \includegraphics[width=10pt]{images_presentation/moyen.png}\tabularnewline
\hline
Réseau &\includegraphics[width=10pt]{images_presentation/question.png}& \includegraphics[width=10pt]{images_presentation/moyen.png} & \includegraphics[width=10pt]{images_presentation/ok.png} & \includegraphics[width=10pt]{images_presentation/question.png}\tabularnewline
\hline
Sécurité & \includegraphics[width=10pt]{images_presentation/question.png}& \includegraphics[width=10pt]{images_presentation/ok.png} &\includegraphics[width=10pt]{images_presentation/ok.png} & \includegraphics[width=10pt]{images_presentation/ok.png}\includegraphics[width=10pt]{images_presentation/ok.png}\tabularnewline
\hline
Simplicité & \includegraphics[width=10pt]{images_presentation/ok.png}& \includegraphics[width=10pt]{images_presentation/moyen.png} & \includegraphics[width=10pt]{images_presentation/ok.png} & \includegraphics[width=10pt]{images_presentation/question.png}\tabularnewline
\hline
Flexibilité & \includegraphics[width=10pt]{images_presentation/question.png}& \includegraphics[width=10pt]{images_presentation/ok.png}& \includegraphics[width=10pt]{images_presentation/moyen.png}& \includegraphics[width=10pt]{images_presentation/question.png}\tabularnewline
\hline
\end{tabular}
\end{center}
\newline
*OXM : OpenXenManager
\end{frame}

%\includegraphics[width=10pt]{images_presentation/moyen.png}
%\includegraphics[width=10pt]{images_presentation/question.png}
%\includegraphics[width=10pt]{images_presentation/bad.png}
%\includegraphics[width=10pt]{images_presentation/ok.png}

\subsection{Problèmes rencontrés et bénéfices}
\begin{frame}{Problèmes rencontrés}
  Et c'est facile.
\end{frame}

\begin{frame}{Bénéfices}
  Et c'est facile.
\end{frame}

\end{document}
