%\documentclass{beamer}

%\usepackage[utf8x]{inputenc}
%\usepackage{default}

%\begin{document}
\begin{frame}
  \begin{center}
  % \huge{Archipel}
   \includegraphics[width=200pt]{images_presentation/logo_archipel.png}
  \end{center}
\end{frame}

\begin{frame}{Architectures}
\begin{itemize}
 \item avec un site
\end{itemize}
\begin{center}
  \includegraphics[width=250pt]{images_presentation/archipel.png}
\end{center}
\end{frame}

\begin{frame}{Architectures}
\begin{itemize}
 \item avec plusieurs sites
\end{itemize}
\begin{center}
\includegraphics[width=200pt]{images_presentation/archipel1.png}
\end{center}
\end{frame}

\begin{frame}{Architecture interne}
\begin{center}
\includegraphics[width=200pt]{images_presentation/intern.png}
\end{center}
\end{frame}

\begin{frame}{Fonctionalitées}
\begin{itemize}
\item Un système de module qui permet d'apporter de nouvelles fonctions
\pause
\item La plus part des opérations de bases sont disponibles : définition d'une nouvelle VM, manipulations du réseau et du stockage,
accès à la console VNC, gestions des snapshots, etc... 
Les opérations de migration sont également prises en charge
\pause
\item Reporting sur l'état de l'hyperviseur,VMCast, planifications de taches, gestions des droits des
utilisateurs, création d'une machine avec load balancing sur les serveurs
\pause
\item Haute disponibilité
\end{itemize}
\end{frame}
\begin{frame}{Problèmes rencontrés}
 \begin{alertblock}{Problèmes}
   \begin{enumerate}
     \item Ajout des différents noeuds
       \pause
     \item Connexions ssh
       \pause
     \item Création d'une nouvelle machine
       \pause
   \end{enumerate}
 \end{alertblock}
\pause
 \begin{exampleblock}{Solutions}
   \begin{enumerate}
     \item Script qui ajoute les noeuds
       \pause
     \item Réplication des clefs
       \pause
     \item Correction d'un bug inhérent à qemu
   \end{enumerate}
 \end{exampleblock}
\end{frame}

